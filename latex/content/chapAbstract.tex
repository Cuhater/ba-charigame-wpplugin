
\chapter*{Kurzfassung/\emph{Abstract}}
\label{chap:abstract}
Diese Arbeit beschreibt die Weiterentwicklung des WordPress-Plugins \textit{Charigame}, einer Plattform für gamifizierte Spendenaktionen.
Ausgangspunkt war eine historisch gewachsene Codebasis mit eingeschränkter Editierbarkeit, uneinheitlichen Strukturen und Sicherheitslücken.

Auf Basis der theoretischen Grundlagen zu WordPress, Gutenberg-Editor und Security-Standards wurde eine modulare Pluginarchitektur entworfen.
Der Ansatz ist objektorientiert, trennt Verantwortlichkeiten und bindet den Gutenberg Editor für die Bearbeitung im Backend ein.
Zentrale Komponenten sind konzipiert und entwickelt worden darunter ein Login Handler, ein Donation Manager und ein Color Manager.

In der Umsetzung wurde ACF Pro durch Carbon Fields ersetzt. Für das Plugin entwickelte Gutenberg-Blöcke machen die Landingpage-Inhalte im Editor vollständig bearbeitbar.
PHPCS stellt die Einhaltung der WordPress Coding Standards sicher. Sicherheitsmaßnahmen wie Nonce-Validierung, Validierung/Sanitizing/Escaping und Session-Management wurden konsequent umgesetzt.
Ein konsistentes Design-System auf Basis von Tailwind/shadcn verbessern Performance und UX.

Die Ergebnisse sind eine deutlich erhöhte Wartbarkeit, eine klare Strukturierung der Verantwortlichkeiten im Code, höhere Sicherheit und eine verbesserte Redaktionserfahrung im Backend.
Ein neu gestaltetes Dashboard fasst Kennzahlen zusammen und erleichtert die Kampagnensteuerung.

Der Ausblick umfasst tiefergehende Analytics mit Exportfunktionen, Testautomatisierung, Internationalisierung sowie zusätzliche Spieltypen und Integrationen.
Damit zeigt die Arbeit, wie sich bestehende WordPress-Plugins mit aktuellen Entwicklungspraktiken nachhaltig weiterentwickeln lassen.



