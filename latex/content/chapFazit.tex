\chapter{Fazit und Ausblick}
\label{ch:fazit-und-ausblick}

Ziel dieser Arbeit war die technische Weiterentwicklung des WordPress-Plugins Charigame.
Im Detail galt es das Plugin zu einer modularen, wartbaren und redaktionell effizient nutzbaren Lösung auszubauen.
Ausgangspunkt war eine historisch gewachsene Codebasis mit inkonsistenten Strukturen, starker Abhängigkeit von ACF Pro sowie eingeschränkter einer Editierbarkeit von Inhalten.
Aufbauend auf den theoretischen Grundlagen, der Analyse des Projektkontexts und der konzeptionellen Ausarbeitung wurden in der Implementierung konkrete Maßnahmen umgesetzt.
Diese Maßnahmen haben die Architektur weiterentwickelt und zentrale Funktionalitäten modernisiert.

\section{Zusammenfassung der Ergebnisse}
Die Weiterentwicklung führte zu Verbesserungen in mehreren Bereichen.
Im Aspekt der Architektur und Wartbarkeit wurde eine klare, modulare Plugin-Architektur mit objektorientierten Komponenten eingeführt.
Die Komponenten darunter Login-Handler, Color Manager und Donation Manager haben Funktionalitäten, die zuvor mit dem Frontend gekoppelt waren getrennt.
Dadurch konnten klare Verantwortlichkeiten gesetzt werden und Abhängigkeiten wurden reduziert und die Erweiterbarkeit verbessert.

Das Entfernen externer Abhängigkeiten erfolgte durch die Migration von ACF Pro zu Carbon Fields, wodurch eine kostenpflichtige, externe Abhängigkeit entfernt wurde.
Bei der Gutenberg-Integration wurden eigene Gutenberg-Blöcke für die bestehenden Sektionen der Landingpage entwickelt.
Somit sind Inhalte der Landingpage direkt im Editor bearbeitbar, was Konsistenz, Performance und redaktionelle Freiheit erhöht.

Standardkonformität und Sicherheit wurden durch systematische Anwendung der WordPress Coding Standards (PHPCS) gewährleistet.
Hier wurde konsequentes Input-Sanitizing und Output-Escaping sowie Nonce-Validierung relevanter AJAX-Endpunkte eingesetzt.

Der Datenzugriff und die Integration sind durch Freigabe relevanter Eigenschaften in der REST-API zur Editor-Integration vereinheitlicht.
Ferner wurde ein Caching von Kampagnenstatistiken eingeführt und die bestehenden Custom Post Types überarbeitet.

Im Bereich Administration und UX entstand ein modernisiertes Dashboard mit wiederverwendbaren Admin-Komponenten, konsistenter UI und verbesserter Übersichtlichkeit für Kampagnensteuerung und Inhalte.

\section{Einordnung der Zielerreichung}
Die formulierten Zielsetzungen wurden erreicht.
Durch die modulare Architektur, klar definierte Klassen und einheitliche Namensgebung ist der Code übersichtlicher, testbarer und nachhaltiger wartbar.
Mit PHPCS und der Ausrichtung an WordPress-Guidelines wurden Best Practices konsistent angewendet.
Die Gutenberg-Blöcke erlauben die Pflege aller wesentlichen Landingpage-Bereiche ohne Codeanpassungen.
Die Ablösung von ACF Pro sowie die Öffnung zur REST-API stärken Zukunfts- und Integrationsfähigkeit.

\section{Reflexion und Grenzen}
Die Arbeit zeigt, dass die Verbindung aus WordPress-Standards, Block-Editor und modularer Architektur ein tragfähiges Fundament für ein produktives Plugin bildet.
Die kohärente Codebasis reduziert Streu-Logik und Altlasten, schafft klarere Verantwortlichkeiten und bessere Testbarkeit.
Die Redaktionserfahrung verbessert sich durch direkte Bearbeitung im Editor, konsistente Vorschau und geringere Abhängigkeit von Entwicklern bei Content-Änderungen.
Durchgängiges Sanitizing und Escaping, Nonces und Caching erhöhen Robustheit und Performance.


Gleichzeitig bestehen Grenzen.
Das Dashboard wurde modernisiert, bietet aber Potenzial für tiefere Auswertungen wie Visualisierungen, Segmentierungen und Exporte.
Unit- und Integrationstests sowie CI/CD-Pipelines sind auszubauen, um Release-Sicherheit zu erhöhen.
Die UI ist konsistent, doch ein systematisch dokumentiertes Designsystem und eine umfassende Barrierefreiheitsprüfung stehen noch aus.

\section{Ausblick}
Auf Basis der Ergebnisse ergeben sich folgende priorisierbare Weiterentwicklungen:

Dashboard und Analytics: KPI-Visualisierungen, Filter- und Segmentierungslogik, CSV/Excel-Export, Benachrichtigungen und Drill-downs erweitern die Auswertungsmöglichkeiten.

Qualitätssicherung: Unit- und Integrationstests für PHP und JavaScript, E2E-Tests für Block-Editor-Flows sowie die Einrichtung einer CI/CD-Pipeline verbessern die Release-Sicherheit.

Berechtigungen und Sicherheit: Feingranulare Capabilities für Kampagnen-, Spiel- und E-Mail-Management sowie Audit-Logging erhöhen die Kontrollmöglichkeiten.

Internationalisierung: Systematische i18n/L10n der Blöcke, Admin-Views und E-Mail-Templates erweitert die Einsatzmöglichkeiten.

Performance: Erweiterte Objekt- und Transient-Caches, Optimierung der Block-Renderpfade, Asset-Bündelung und Lazy Loading steigern die Effizienz.

Barrierefreiheit: WCAG-orientierte Überarbeitung, Komponentenbibliothek mit dokumentierten Patterns, Theme-Kompatibilität und Block Patterns verbessern die Zugänglichkeit.

Funktionalität: Weitere Spieltypen, erweiterte Game-Settings als wiederverwendbare Presets und A/B-Testing von Spielmechaniken erweitern den Funktionsumfang.

Integration: Datenschutzkonforme Tracking-Konzepte, optionale Anbindung an externe CRMs jenseits projektspezifischer Integrationen und stabile, versionierte REST-Endpoints verbessern die Integrationsmöglichkeiten.

Veröffentlichung: Vorbereitung für eine Veröffentlichung im WordPress-Ökosystem durch Readme, Lizenz, Update-Strategie und Support-Prozess.

\section{Schlussbemerkung}
Mit der vorliegenden Weiterentwicklung wurde Charigame auf eine belastbare, erweiterbare und standardkonforme Basis gestellt.
Die redaktionelle Arbeit im Gutenberg-Editor, die konsolidierte Architektur sowie sicherheits- und qualitätsorientierte Maßnahmen schaffen unmittelbar Mehrwert im Betrieb.
Die skizzierten nächsten Schritte bieten einen klaren Pfad, um das Plugin hinsichtlich Analytics, Automatisierung, Designsystem und Funktionsumfang weiter zu professionalisieren und strategisch auszubauen.

