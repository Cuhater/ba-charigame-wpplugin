\chapter{Theoretische Grundlagen}
\section{WordPress als Content-Management-System}
\section{Plugin-Entwicklung mit WordPress}
\section{Gutenberg-Editor: Konzept und technische Grundlagen}
\section{Gamification im Kontext digitaler Anwendungen}
\section{Überblick über Spendenverteilung und Charity-Plattformen}
\label{chap:formal}
%
In diesem Kapitel finden Sie grundlegende Hinweise zum formalen Aufbau Ihrer Arbeit.
%
\textbf{Reihenfolge}
\label{sec:aufbau}
Eine wissenschaftliche Arbeit besteht in der Regel aus den folgenden Teilen:
%
\begin{enumerate}
 \item Deckblatt
 \item Kurzfassung/Abstract (optional)
 \item Inhaltsverzeichnis
 \item Abbildungs- und Tabellenverzeichnis (auch am Ende üblich)
 \item Abkürzungsverzeichnis (auch am Ende üblich)
 \item Einleitung
 \item Hauptteil
 \item Zusammenfassung/Fazit
 \item Literaturverzeichnis
 \item Anhänge (optional)
 \item Erklärung
\end{enumerate}
%
%
\textbf{Deckblatt}
%Die Gestaltung des Deckblatts folgt den visuellen Vorgaben für Publikationen der TH Köln.
%\par
Das Deckblatt beinhaltet: Titel der Arbeit, Art der Arbeit, Verfasser*in, Matrikelnummer, Abgabetermin, Betreuer*in sowie Zweitgutachter*in. Das Deckblatt wird bei Arbeiten, die länger sind als~15 Seiten, bei der Seitenanzahl zwar mitgezählt, jedoch nicht nummeriert.
%
%
\textbf{Inhaltsverzeichnis}
\label{sec:listOfContents}
Wir empfehlen eine Dezimalgliederung wie in diesem Dokument angelegt. Werden innerhalb eines Kapitels Unterüberschriften verwendet, müssen mindestens zwei vorhanden sein: wo ein~2.1 ist, muss es ein~2.2 geben.
\par
Das Inhaltsverzeichnis enthält immer die Seitenangaben zu den aufgelisteten Gliederungspunkten; es wird dabei aber selbst nicht im Inhaltsverzeichnis aufgelistet. Die Seiten, die das Inhaltsverzeichnis selbst einnimmt, können römisch gezählt werden.
%Mehr hierzu in Abschnitt~\cref{}.
\par
Für eine Abschlussarbeit ist eine Gliederungstiefe von wenigstens drei Ebenen üblich. In der Regel werden nur bis zu vier Ebenen vorne im Inhaltsverzeichnis abgebildet. Hier sollten Sie aber unbedingt die Gepflogenheiten in Ihrem Fach berücksichtigen und ggf. in Erfahrung bringen.
%\par
%In dieser Word-Vorlage wird das Inhaltsverzeichnis für die Überschriftenebenen 1 bis 3 automatisch generiert (Rechtsklick auf das Inhaltsverzeichnis > Felder aktualisieren > Ganzes Verzeichnis).
%
%
\textbf{Abbildungsverzeichnis und Tabellenverzeichnis}
Abbildungen und Tabellen werden in entsprechenden Verzeichnissen gelistet. In dieser Vorlage erscheinen sie direkt nach dem Inhaltsverzeichnis. Dann können die entsprechenden Seiten römisch gezählt werden. Die Verzeichnisse können jedoch auch am Ende der Arbeit vor oder hinter dem Literaturverzeichnis stehen. Dann werden sie regulär mit Seitenzahlen versehen.
%Verzeichnisüberschriften (z. B. Abbildungsverzeichnis) werden nie nummeriert (Formatvorlage Überschrift 1 unnummeriert verwenden).
%
%
\textbf{Abkürzungsverzeichnis}
Die Zahl der Abkürzungen sollte übersichtlich bleiben. Das Abkürzungsverzeichnis enthält lediglich wichtige fachspezifischen Abkürzungen in alphabetischer Reihenfolge, insbesondere Abkürzungen von Organisationen, Verbänden oder Gesetzen. Gängige Abkürzungen wie \enquote{u.\,a.}, \enquote{z.\,B.}, \enquote{etc.} werden nicht aufgenommen.
\par
Zur technischen Umsetzung mit \LaTeX{} vergleiche auch Abschnitt~\ref{sec:template}.
%
%
\textbf{Literaturverzeichnis}
Das Literaturverzeichnis wird alphabetisch nach Autorennamen geordnet. Es enthält alle im Text zitierten Quellen~--~und nur diese. Mehrere Schriften einer Person werden nach Erscheinungsjahr geordnet. Schriften derselben Person aus einem Erscheinungsjahr müssen Sie selbst unterscheidbar machen. In den Ingenieurwissenschaften wird zusätzlich häufig ein Nummern- oder Autorenkürzel dem Namen in eckigen Klammern voran-gestellt. Mehr hierzu und weitere wichtige Regeln des Zitierens lernen Sie in den E-Learning-Kursen des Schreibzentrums\footnote{\href{https://ilu.th-koeln.de/goto.php?target=cat\_52109\&client\_id=thkilu}{https://ilu.th-koeln.de/goto.php?target=cat\_52109\&client\_id=thkilu}} kennen.
\par
Zur Verwaltung der verwendeten Literatur eigenen sich entsprechende Softwaretools wie Citavi oder Zotero, die mit verschiedenen Textverarbeitungsprogrammen kompatibel sind.
%
%
\textbf{Rechtschreibung, Grammatik}
Achten Sie bei der Abgabe Ihrer Arbeit auf ein einwandfreies Deutsch bzw. Englisch. Wenn Fehler die Lesbarkeit beeinträchtigen, kann sich dies durchaus negativ auf die Note auswirken. Nutzen Sie daher unbedingt die Rechtschreibprüfung Ihres Textverarbeitungsprogramms, auch wenn diese nicht alle Fehler erkennt.
%In Word können Sie diese unter Datei > Optionen > Dokumentenprüfung bearbeiten sowie ein- und ausschalten.
\par
Für alle, die sich bei diesem Thema unsicher fühlen, empfehlen wir die E-Learning-Kurse des Schreibzentrums\footnote{\href{https://ilu.th-koeln.de/goto.php?target=cat\_52109\&client\_id=thkilu}{https://ilu.th-koeln.de/goto.php?target=cat\_52109\&client\_id=thkilu}}. Wenden Sie sich ggf. auch an die Beauftragte für Studierende mit Beeinträchtigung\footnote{\href{https://www.th-koeln.de/studium/studieren-mit-beeintraechtigung\_169.php}{https://www.th-koeln.de/studium/studieren-mit-beeintraechtigung\_169.php}}.
%
%
\textbf{Umfang der Arbeit}
Alle Fächer nennen verbindliche Angaben zu Unter- und Obergrenzen, die in der Regel eingehalten werden müssen. Verzeichnisse und Anhänge werden dabei in aller Regel nicht mitgezählt. In Einzelfällen~--~insbesondere bei empirischen Arbeiten~--~können abweichende Vereinbarungen mit der Betreuungsperson getroffen werden.