\chapter{Implementierung}
\label{chap:literature}
Auf Basis der in Kapitel 4 entwickelten Konzeption folgt in diesem Kapitel die praktische Umsetzung. Dabei werden die einzelnen Komponenten des Systems detailliert beschrieben und ihre Implementierung im Rahmen des WordPress-Plugins erläutert. Neben technischen Aspekten wie der Strukturierung von Backend und Frontend sowie der Integration des Gutenberg-Editors wird auch auf die konkrete Realisierung der funktionalen Anforderungen eingegangen.
\section{Entwicklungsumgebung und Tools}
Für die Umsetzung des Projekts wurden verschiedene Tools eingesetzt.
Als Entwicklungsumgebung kam PHPStorm zum Einsatz, ergänzt durch PHPCS zur Einhaltung der WordPress-Codestandards.
Google Chrome diente als Browser, während die lokale Entwicklung mit WP Local erfolgte.
Zur Verwaltung von Paketen und Abhängigkeiten wurde NPM verwendet und die Versionierung des Projekts erfolgte über GitHub.

\section{Projektsetup}
Im ersten Schritt der Implementierung wurde eine lokale Entwicklungsumgebung mit WP Local aufgesetzt.
Hierzu nginx Server mit PHP Version 8.4.10 und einer MySQL 8.0.35 Database.
Wordpress kommt in Version 6.8.2 zum Einsatz.
Die einstellung normal keine weiteren Anpassungen getroffen.
Keine weiteren Plugins in WP hinterlegt.

Einrichtung des Projekts auf GitHub / Versionskontrolle
Github neuer Branch des bestehenden Projekts auf dem internen Elancer GitHub erstellt
Projekt importiert ansonsten erstmal nichts

\section{Plugin-Struktur und Architektur}
Vorgehen der Implementierung
Architektur
Plugin bereinigt von altlasten. hierzu zählen diverse assets und abhängigkeiten die nicht mehr benötigt wurden.
Carbon fields hinterlegt als klassenaufruf instanziiert den boot
Dann alle get field aufrufe identifiziert und durch carbon fields ersetzt
Dann alle CPTs neu aufgebaut. Einheitlichen ductus von Charigame realisiert.
Namespace angepasst und klassen bezeichnungen und datenbank bezeichnungen angeeglichen
Verzeichnisstrutkur aufgebaut - includes,
Überblick über die Dateien und Verzeichnisse

Custom Post Types, Shortcodes, Blocks

Verbindungen zwischen Frontend und Backend

\section{Implementierte Features}

Beschreibe die konkreten Funktionen, z. B.:

Frontend-Darstellung

Backend-Verwaltung

Gamification-Elemente

Gutenberg-Blöcke

Eventuell nach Funktionsblöcken oder Modulen gliedern

