\documentclass[
	ngerman,
	%twoside,
	BCOR=8mm,
	headings=normal,
	parskip=half,
	headsepline,
	automark,
	listof=totoc,
	bibliography=totoc,
	%,captions=tableabove
	%draft
]{scrreprt}
\usepackage{graphicx}
%
%
%%%%%%%%%%%%%%%%%%%%%%%%%%%%%%%%%%%%%%%%%%%%%%%%%%%%%%%%%%%%%%%%%%%%%%%%%%%%%%%
% Pakete laden
%%%%%%%%%%%%%%%%%%%%%%%%%%%%%%%%%%%%%%%%%%%%%%%%%%%%%%%%%%%%%%%%%%%%%%%%%%%%%%%
%
\usepackage{ifluatex}
\usepackage{babel}
%
\ifluatex
 % LuaLaTeX
 \usepackage{fontspec}
 \usepackage{selnolig}
\else
 % PdfLaTeX
 \usepackage[T1]{fontenc}
\fi
%
\usepackage{csquotes}
\usepackage{scrlayer-scrpage}
\usepackage{microtype}
%
%\usepackage{ziffer}% optional
%\usepackage[locale=DE]{siunitx}% optional
%
\usepackage{tikz}
\usepackage{pgfplots}
%
\usepackage[style=ieee,backend=biber,language=german]{biblatex}
%
\usepackage{amsmath}
\usepackage{amssymb}
\usepackage{epigraph}
\usepackage{courier}
\usepackage{longtable}
\usepackage{array}
%
% compile fix
\usepackage{lmodern}

\usepackage{hyperref}
%
%=== wichtig, dass folgende Pakete NACH hyperref geladen werden ===============
\usepackage{scrhack}% Um Warnung bzgl. \float@addtolists im listings-Paket (s.u.) zu vermeiden
\usepackage{listings}
%
\usepackage[nameinlink]{cleveref}
\usepackage[all]{hypcap}
\usepackage[
	toc,
	symbols,
	acronyms,
]{glossaries}
%
\input{definitions}
%
%%%%%%%%%%%%%%%%%%%%%%%%%%%%%%%%%%%%%%%%%%%%%%%%%%%%%%%%%%%%%%%%%%%%%%%%%%%%%%%
% Begriffe für glossaries definieren
%%%%%%%%%%%%%%%%%%%%%%%%%%%%%%%%%%%%%%%%%%%%%%%%%%%%%%%%%%%%%%%%%%%%%%%%%%%%%%%

%=== Glossar ==================================================================
\newglossaryentry{plugin}{
	name={Plugin},
	description={Softwareerweiterung bestehend aus Codepaketen, die die
	Kernfunktionalität von WordPress erweitern. Plugins bestehen
	aus PHP-Code und können weitere Assets wie Bilder, CSS und
	JavaScript enthalten. \\(eigene Übersetzung nach \cite{wordpress2024plugin})}
}
\newglossaryentry{wdr}{
	name={Wordpress Developer Resources},
	description={Ein online Handbuch, das Entwicklungsthemen in Wordpress abdeckt. Es wird kontinuierlich weiterentwickelt und gibt Entwicklern Definitionen, Coding Standards, Wordpress Core Ressources, Block Editor Ressources, gängige APIs, Beispiele und Tutorials an die Hand}
}
\newglossaryentry{csr}{
    name={Corporate Social Responsibility},
    description={Unter "Corporate Social Responsibility" (CSR) ist die gesellschaftliche Verantwortung von Unternehmen im Sinne eines nachhaltigen Wirtschaftens zu verstehen.\cite{BMAS2022_CSRGrundlagen}}
}
\newglossaryentry{dog}{
	name={Hund},
	description={Behaartes, vierbeiniges Säugetier. Bester Freund des
	Menschen}
}

%=== Abkürzungen ==============================================================
\newacronym{svm}{SVM}{support vector machine}

\newacronym{acf}{ACF PRO}{Advanced Custom Fields Pro}
\newacronym{ajax}{AJAX}{Asynchronous JavaScript and XML}
\newacronym{api}{API}{Application Programming Interface}
\newacronym{b2b}{B2B}{Business-to-Business}
\newacronym{b2c}{B2C}{Business-to-Consumer}
\newacronym{cdn}{CDN}{Content Delivery Network}
\newacronym{ci}{CI}{Corporate Identity}
\newacronym{cli}{CLI}{Command Line Interface}
\newacronym{cms}{CMS}{Content-Management-System}
\newacronym{crm}{CRM}{Customer-Relationship-Management}
\newacronym{csr}{CSR}{Corporate Social Responsibility}
\newacronym{csrf}{CSRF}{Cross-Site Request Forgery}
\newacronym{css}{CSS}{Cascading Style Sheets}
\newacronym{cta}{CTA}{Call to Action}
\newacronym{gnu}{GNU}{GNU's Not Unix}
\newacronym{gpl}{GPL}{General Public License}
\newacronym{gsap}{GSAP}{GreenSock Animation Platform}
\newacronym{html}{HTML}{Hypertext Markup Language}
\newacronym{http}{HTTP}{Hypertext Transfer Protocol}
\newacronym{js}{JS}{JavaScript}
\newacronym{json}{JSON}{JavaScript Object Notation}
\newacronym{kpi}{KPI}{Key Performance Indicator}
\newacronym{mp4}{MP4}{Moving Picture Experts Group 4 Part 14}
\newacronym{mysql}{MySQL}{My Structured Query Language}
\newacronym{npm}{NPM}{Node Package Manager}
\newacronym{orm}{ORM}{Object-Relational Mapping}
\newacronym{php}{PHP}{Hypertext Preprocessor}
\newacronym{rest}{REST}{Representational State Transfer}
\newacronym{smtp}{SMTP}{Simple Mail Transfer Protocol}
\newacronym{ssl}{SSL}{Secure Sockets Layer}
\newacronym{sql}{SQL}{Structured Query Language}
\newacronym{tls}{TLS}{Transport Layer Security}
\newacronym{uri}{URI}{Uniform Resource Identifier}
\newacronym{url}{URL}{Uniform Resource Locator}
\newacronym{wp}{WP}{WordPress}
\newacronym{xml}{XML}{Extensible Markup Language}
\newacronym{xsl}{XSL}{Extensible Stylesheet Language}
\newacronym{xss}{XSS}{Cross-Site-Scripting}

%=== Symbole ==================================================================
\newglossaryentry{sym:force}{
	name=\ensuremath{\vec{F}},
	description={Kraft, vektorielle Größe},
	type=symbols,
}

%
\makeglossaries
%
\graphicspath{{images/}}
%
\addbibresource{references.bib}

\nocite{*} % listet alle Quellen (auch die nicht zitierten!)
%
%=== für schnelleres kopilieren alle ungeänderten Dateien auskommentieren =====
\includeonly{
    content/chapDeclaration,
    content/chapAbstract,
    content/chapEinleitung,
	content/chapBasics,
    content/chapContext,
    content/chapConceptDesign,
    content/chapImplement,
    content/chapFazit,
}
%
%
%==============================================================================
%
\begin{document}
%
\pdfbookmark[0]{Titelseite}{titel}
\begin{titlepage}
%
\sffamily% Umschalten auf serifenlose Schrift
%
\begin{center}
\begin{tikzpicture}
 \fill[THRed] (0, 0) rectangle (\textwidth/3, 3pt);
 \fill[THOrange] (\textwidth/3, 0) rectangle (2*\textwidth/3, 3pt);
 \fill[THPurple] (2*\textwidth/3, 0) rectangle (\textwidth, 3pt);
\end{tikzpicture}
\end{center}
%
\vfill
%
\begin{huge}
    Weiterentwicklung eines WordPress-Plugins zur gamifizierten Spendenverteilung:
    Gutenberg-Integration und Plugin-Architektur am Beispiel von ``Charigame'' in Kooperation mit der elancer-team GmbH\\[10mm]
\end{huge}
%
Bachelorarbeit zur Erlangung des akademischen Grades\newline
\emph{Bachelor of Science}\newline
im Studiengang Medieninformatik\newline
an der Fakultät für Informatik und Ingenieurwissenschaften\newline
der Technischen Hochschule Köln
%
\vfill
%
\begin{tabular}{@{}ll}
vorgelegt von: & Christian Krenn\\
               & chrsitian.krenn@smail.th-koeln.de\\[5mm]
eingereicht bei:   & Prof. Dr. Hoai Viet Nguyen\\
Zweitgutachter*in: & Tobias Derksen
\end{tabular}	
%
\\[10mm]
%
Gummersbach, 03.09.2025%
%
\rmfamily% Umschalten auf Standard-Schrift mit Serifen
%
\end{titlepage}
\cleardoublepage
\pagenumbering{Roman}
\chapter*{Kurzfassung/\emph{Abstract}}
\label{chap:abstract}
%


\cleardoublepage
\pdfbookmark[0]{Inhaltsverzeichnis}{toc}
\tableofcontents
\listoftables
\listoffigures
\printglossary[title=Glossar,toctitle=Glossar]
\printglossary[type=\acronymtype, title={Abkürzungsverzeichnis}]
\printglossary[type=symbols, title={Symbolverzeichnis}]
%
\cleardoublepage
\KOMAoptions{open=right}
\pagenumbering{arabic}
\chapter{Einleitung}
%\begin{quote}
%    ``Simplicity is a great virtue but it requires hard work to achieve it and education to appreciate it.
%    And to make matters worse: complexity sells better.''\cite{dijkstra1982}
%\end{quote}

%\footnote{\href{https://www.th-koeln.de/schreibzentrum}{https://www.th-koeln.de/schreibzentrum}}
%\\\\
%Technische Einfachheit bildet ein zentrales Qualitätsmerkmal moderner Softwareentwicklung. Sie fördert Wartbarkeit, Verständlichkeit und langfristige Erweiterbarkeit.

%\\\\
%Technische Einfachheit bildet ein zentrales Qualitätsmerkmal moderner
%Softwareentwicklung \cite{martin2008clean,fowler2018refactoring}. Sie
%fördert Wartbarkeit, Verständlichkeit und langfristige Erweiterbarkeit,
%was auch in internationalen Standards wie der ISO/IEC 25010 als wichtiges
%Qualitätskriterium anerkannt wird \cite{iso25010}.
%\\
%Nach \cite{martin2008clean} ist Einfachheit eines der fundamentalen
%Prinzipien für sauberen Code, während \cite{fowler2018refactoring}
%betont, dass einfache Strukturen die Basis für erfolgreiche
%Refaktorierung bilden.
%\\
%Um dies zu erreichen, erfordert es gezielte Architekturentscheidungen und eine klare Reduktion unnötiger Abhängigkeiten.
%\\
%Das WordPress-Plugin Charigame der elancer-team GmbH bietet Unternehmen eine digitale Möglichkeit, ihr gesellschaftliches Engagement spielbasiert zu kommunizieren.
%\\
%Nutzer nehmen an interaktiven Kampagnen teil, etwa in Form von Memory- oder Geschicklichkeitsspielen, und beeinflussen damit die Verteilung einer vom Unternehmen bereitgestellten Spendensumme auf soziale Projekte. Dieser partizipative Ansatz verbindet unternehmerische CSR-Maßnahmen mit spielerischer Nutzerinteraktion.
%Die vorliegende Arbeit entwickelt diesen Ansatz technisch weiter.
%Sie ersetzt die bestehende Konfigurationslogik, die derzeit auf ACF PRO basiert, durch eine eigenständige, modulare Struktur.
%Zudem entsteht eine vollständige Integration in den Gutenberg-Editor durch individuell entwickelte Blöcke.
%Die neue Architektur verbessert die Wartbarkeit, erhöht die Flexibilität für Entwickler und erleichtert die redaktionelle Nutzung im WordPress-Backend.%Ziel der vorliegenden Arbeit ist es, diesen technischen Ansatz weiterzuentwickeln, um sowohl die Funktionalität als auch die Benutzerfreundlichkeit des Plugins zu verbessern.

Das WordPress-\gls{plugin} Charigame der Agentur elancer-team GmbH bietet Unternehmen die Möglichkeit, Spendenaktionen über gamifizierte Inhalte online anzubieten.
Die Nutzer beteiligen sich an interaktiven Kampagnen, die als Memory- oder Geschicklichkeitsspiele gestaltet sind.
Je nach Interaktion der Kunden wird bestimmt, welche sozialen Projekte aus einem vom Unternehmen bereitgestellten Spendentopf unterstützt werden.
Hierdurch werden unternehmerische \gls{csr}-Maßnahmen mit spielerischer Nutzerbeteiligung kombiniert.

\epigraph{``Clean code always looks like it was written by someone who cares."}{--- \textup{Robert C. Martin}}

Diese Philosophie bildet die Grundlage für die Analyse und Weiterentwicklung der bestehenden Struktur.
Für diese Weiterentwicklung wird ein Konzept erarbeitet, das auf den gesammelten theoretischen Erkenntnissen aufbaut.
Die Ausarbeitung sieht vor das System Charigame wartungsfreundlicher zu gestalten und Entwicklern mehr Flexibilität bei Anpassungen zu bieten.
Ferner gilt es Redakteuren eine intuitive Nutzung im Wordpress-Backend zu ermöglichen.

\section{Problemstellung und Motivation}

Die deskriptive Codebasis des WordPress-Plugins Charigame weist einen strukturellen Optimierungsbedarf auf, der auf die Entstehungsgeschichte des Plugins zurückgehen.
Ursprünglich wurde das Plugin als internes Projekt der elancer-team GmbH entwickelt und später gezielt auf die Bedürfnisse eines spezifischen Kunden angepasst.

Während dieser kundenspezifischen Entwicklung entstanden viele Funktionen, die stark auf diesen besonderen Anwendungsfall zugeschnitten sind.
Eine übergreifende, flexibel erweiterbare Architektur wurde hierbei nicht berücksichtigt.
Das Hauptproblem liegt darin, dass die aktuelle Implementierung gängige WordPress-Standards nur unzureichend berücksichtigt.
Aufgrund des initial gesetzten Zeitrahmens entstand eine Codebasis, die zwar grundsätzlich funktioniert, jedoch nicht etablierten Best Practices der Plugin-Entwicklung entspricht.
Das zeigt sich beispielsweise in mangelnder Modularität, lückenhafter Dokumentation und einer Struktur, die weitere Anpassungen erschwert.

\textbf{Motivation}
Die Motivation für diese Arbeit ergibt sich aus drei zentralen Aspekten:
\begin{itemize}
    \item \texttt{Eigene Weiterentwicklung}: Die Neuausrichtung des Plugins bietet die Chance, sich intensiv mit modernen WordPress-Entwicklungsmethoden zu beschäftigen. Besonders im Bereich modularer Architektur und Block-Entwicklung im Gutenberg-Umfeld.

    \item \texttt{Technologischer Fortschritt}: Die Integration des Gutenberg-Editors als zukunftsfähiger, visueller Backend-Builder bringt das Plugin auf den neuesten Stand der WordPress-Technologie. Das verbessert nicht nur die redaktionelle Nutzererfahrung, sondern schafft auch die Basis für eine nachhaltigere technische Grundlage.

    \item \texttt{Mehrwert für die Agentur}: Eine klar strukturierte, wartbare Codebasis bringt konkrete Vorteile für die elancer-team GmbH. Sie ermöglicht eine effizientere Teamarbeit, erleichtert den Wissenstransfer und verringert die Abhängigkeit von Einzelpersonen. Langfristig schafft das mehr Flexibilität, Stabilität und Skalierbarkeit im Projekt.
\end{itemize}
Die Weiterentwicklung des Plugins ist somit ein logischer nächster Schritt.


\section{Zielsetzung der Arbeit}

Das Ziel dieser Arbeit ist die technische Weiterentwicklung des bestehenden WordPress-Plugins Charigame mit besonderem Fokus auf Wartbarkeit, Standardkonformität und redaktionelle Nutzbarkeit.
Durch die Anpassung der Architektur soll der Code besser strukturiert und anhand der WordPress-Plugin-Standards gestaltet werden.

Die Arbeit beschäftigt sich mit den theoretischen Grundlagen und passt das Plugin an bewährte WordPress-Standards an.
Zunächst wird der aktuelle Entwicklungsstand gründlich analysiert.
Auf dieser Grundlage erfolgt dann die praktische Umsetzung.
Dabei werden verschiedene Lösungsansätze betrachtet und die getroffenen Entscheidungen transparent erläutert.

\section{Methodisches Vorgehen}

Die Zielerreichung erfolgt praxisnah und methodisch strukturiert.
Damit das gesetzte Ziel erreicht werden kann wird ein methodisches Vorgehen durchgeführt, das sich grob in vier größere Bestandteile gliedern lässt.

\begin{enumerate}
    \item \texttt{Ermitteln des Deskriptive Stands}: Im ersten Schritt wird die aktuelle Plugin-Struktur auf Basis der offiziellen WordPress-Guidelines beleuchtet.
    Hier wird grundlegend eine modulare Architektur verfolgt, damit die langfristige Wartbarkeit und Erweiterbarkeit gegeben ist.

    \item \texttt{Anforderungsanalyse und Konzeption}: Die funktionalen und nicht-funktionalen Anforderungen an das Plugin werden im zweiten Schritt erfasst und dienen als Grundlage für die Anpassung des Systems.

    \item \texttt{Reduktion externer Abhängigkeiten}: Im dritten Schritt soll die bisherige Abhängigkeit von \gls{acf} durch eine eigene Lösung ersetzt werden.
    Dieser Schritt ist wichtig, um den Funktionsumfang lizenzfrei zu halten und zusätzliche Kosten zu vermeiden.

    \item \texttt{Integration des Gutenberg-Editors}: Der letzte Schritt sieht vor eine Block-basierte Verwaltung im Backend zu entwickeln.
    Diese Lösung soll auf dem Gutenberg-Editor basieren und somit dem aktuellen Standards von WordPress entsprechen.
\end{enumerate}

\section{Aufbau der Arbeit}

Die Arbeit beginnt mit der erfassen und sammeln der theoretischen Grundlagen hinsichtlich der Wordpress-Entwicklungsstandards.
Darauf folgt eine Analyse der aktuellen Codebasis und der bestehenden Architektur, wodrauf basierend ein neues technisches Konzept erarbeitet und implementiert wird.
Die Umsetzung wird schließlich im Hinblick auf Struktur, Funktionalität und redaktionelle Bedienbarkeit evaluiert.
Der Aufbau der Arbeit spiegelt diesen Ablauf in aufeinander aufbauenden Kapiteln wider.


%\section{LA LI LU NUR DER MANN IM MOND SCHAUT ZUUUU}
%
%Das vorliegende Dokument kann als Muster und Anleitung für wissenschaftliche Abschlussarbeiten verwendet werden. Es beruht ursprünglich auf einem Leitfaden, den Prof.~Dr.~Stephan Freichel als Prüfungsausschussvorsitzender für die Studiengänge B.\,Sc.~Logistik und M.\,Sc.~\emph{Supply Chain and Operations Management} an der Fakultät für Fahrzeugsysteme und Produktion erstellt hat.
%\par
%Der Text in dieser Vorlage beschreibt allgemeine formale Anforderungen, insbesondere zum Inhaltsverzeichnis, zum Einfügen von Quellenverweisen und zum Erstellen eines Literaturverzeichnisses.
%\par
%Die Vorlage kann fachübergreifend als Musterdatei für Abschlussarbeiten an der TH Köln verwendet werden. Allerdings müssen Sie dann unbedingt klären, ob sie den Konventionen in ihrem Studienfach entspricht.
%\par
%Für die Möglichkeit eines fachunabhängigen Gebrauchs wurde das Dokument von Maria-Anna Worth (i.\,R.) und Susanne Neuzerling (Hochschulreferat Planung und Controlling) inhaltlich überarbeitet und modifiziert. Eine erneute Überarbeitung und Aktualisierung erfolgte durch Andreas Bissels (Schreibzentrum). Frau Katharina Bata hat die Dokumentvorlage in LaTeX erstellt. Zuletzt wurde diese Vorlage 2023 von André Ulrich und Jan Salmen überarbeitet und um diverse Hinweise speziell zur Gestaltung mit \LaTeX{} ergänzt (siehe \cref{chap:Textsatz}).
%\par
%Zur Vorbereitung auf Ihre Abschlussarbeit empfehlen wir Ihnen die Angebote des Schreibzentrums der Kompetenzwerkstatt\footnote{\href{https://www.th-koeln.de/schreibzentrum}{https://www.th-koeln.de/schreibzentrum}}; hierzu gehören sowohl eine Schreibberatung als auch Schreibkurse. Das Schreibzentrum ist Ihre Anlaufstelle an der TH Köln in Fragen rund um das wissenschaftliche Schreiben.
%\par
%Sichern Sie diese Dokumentvorlage bitte zweifach auf Ihrem Rechner: Einmal, um weiterhin auf den hier dargestellten Inhalt zugreifen zu können, und ein zweites Mal, um sie mit Ihrer eigenen Abschlussarbeit zu überschreiben.
%\par
%\emph{Bitte beachten Sie: Die Vorlage ersetzt nicht die spezifischen Vorgaben der jeweiligen Prüfungsausschüsse. Sollte es in Ihrem Fach besondere formale Vorgaben geben, so gelten diese.}\enlargethispage{\baselineskip}
%
%\begin{flushright}
%Köln, August 2023
%\end{flushright}
\chapter{Theoretische Grundlagen}
Dieses Kapitel eruiert die theoretischen Grundlagen für die Ausarbeitung, welche für das Verständnis dieser Arbeit notwendig sind und im praktischen Teil angewendet werden.
\section{WordPress als Content-Management-System}
WordPress ist ein freies, unter der \gls{gnu} \gls{gpl} v2 lizenziertes Content-Management-System und mit einem Marktanteil von 61,1\% das weltweit am häufigsten verwendete CMS. \cite{statista2025cms}
Die Plattform zeichnet sich durch seine flexibilität, erweiterbarkeit und bedienbarkeit aus, was Wordpress zu einer Lösung für simple Webseiten und Blogs bis hin zu komplexe Web-Applikationen macht. \cite{patel2019review}
Die technische Struktur verfolgt ein modulares Paradigma, das eine klare Trennung zwischen Kernfunktionen, Design und Erweiterungen ermöglicht:
\begin{itemize}

 \item Themes: Kontrollieren die Präsentationslogik und das Design der Webseite

 \item Plugins: Erweitern die Funktionalität

 \item Core System: Stellt die grundlegenden Funktionen des CMS bereit

\end{itemize}
Technisch gesehen besteht Wordpress aus einer \gls{php}-basierten Architektur in Verbindung mit der persistenten Speicherlösung \gls{mysql} als relationale Datenbank.
Durch diese weit verbreiteten Technologien ist WordPress mit den meisten gängigen Hosting-Umgebungen kompatibel.
Grundlage hierfür ist ein Webserver, welcher PHP und MySQL unterstützt.
Offiziell ist Apache oder Nginx empfohlen \cite{wordpress2024requirements}.

Darüber hinaus bietet Wordpress eine \gls{rest}-\gls{api} mit der eine entkoppelte Inhaltsverwaltung möglich ist und Inhalte über das \gls{http}-Protokoll in verschiedene Anwendungen und Plattformen integriert werden können.
\newpage
\textbf{WordPress Coding Standards und Code-Qualität}\\
Wordpress Coding Standards dienen als Richtlinien und Best Practises für Entwickler die an Wordpress Projekten arbeiten.
Diese Standards haben sich in der Wordpress Community etabliert und ermöglichen eine konsistente Codestruktur.
Hintergrund ist es die Codebasis für bessere Wartbarkeit und kollaborative Arbeit zu strukturieren.
Für die Ausarbeitung der Weiterentwicklung von Charigame sollen die offiziellen Wordpress Coding Standards beachtet und umgesetzt werden.


\section{Plugin-Entwicklung mit WordPress}

Die offizielle WordPress-Definition besagt: \glqq Plugins sind Codepakete, die die Kernfunktionalität von WordPress erweitern.
WordPress-Plugins bestehen aus PHP-Code und können weitere Assets wie Bilder,
CSS und JavaScript enthalten\grqq{} \cite{wordpress2024plugin} (eigene Übersetzung).
\\
Für die Entwicklung solcher Plugins ist grundlegend ein tiefergehendes Verständnis des Wordpress-CMS vorausgesetzt.
Die Grundlagen sowie weiterführende Themengebiete sollen im Folgenden ermittelt werden.



\subsection{Grundlagen der Plugin-Architektur}

Die Entwicklung von Plugins in WordPress setzt ein Basiswissen der Dateistruktur und des internen Aufbaus des Content-Management-Systems voraus.
Insbesondere ist es erforderlich, die Plugin-bezogenen Dateien korrekt zu strukturieren und im entsprechenden Verzeichnis innerhalb der WordPress-Installation zu platzieren.\\\\
Plugins werden standardmäßig im Verzeichnis \texttt{wp-content/plugins} abgelegt.
Jeder Plugin-Ordner enthält dabei alle notwendigen Dateien zur Funktionalität des jeweiligen Moduls.\\\\
Im folgenden Schaubild ist eine typische, nicht modifizierte Verzeichnisstruktur einer WordPress-Installation dargestellt.
Diese dient als Ausgangspunkt für die Entwicklung und Integration von Plugins:

\begin{figure}[tbh]
 \centering
 \includegraphics[width=0.88\textwidth]{wordpress_ordner_aufbau}
 \caption{Unveränderte WordPress-Verzeichnisstruktur (6.8.2)}
 \label{fig:wordpress-verzeichnis}
\end{figure}
\newpage
Abbildung 2.1 visualisiert, dass sich im wp-content Ordner bereits initial in der Standardkonfiguration von Wordpress das Verzeichnis \texttt{plugins} befindet.
Der \texttt{plugins} Ordner ist die zentrale Destination für alle im Wordpress System erstellten Erweiterungen.
Die nachfolgende Untersuchung berücksichtigt weitere Bestandteile von WordPress-Plugins, die fortlaufend detailliert beschrieben werden.
\\\\\\
\textbf{Aufbau, Plugin-Header und Metadateien}

Für die Entwicklung von WordPress-Plugins empfehlen die offiziellen WordPress Developer Resources eine klare und strukturierte Ordnerorganisation\cite{wordpress2024BestPractices}.
Als Empfohlen wird folgende grundlegende Verzeichnisstruktur vorgeschlagen:
\\\\\\
\begin{lstlisting}[caption={Beispielhafte Plugin-Verzeichnisstruktur in WordPress}, label={lst:plugin-structure}]
/plugin-name
│
├── plugin-name.php
├── uninstall.php
│
├── /languages
├── /includes
│
├── /admin
│   ├── /js
│   ├── /css
│   └── /images
│
└── /public
    ├── /js
    ├── /css
    └── /images
\end{lstlisting}

Die plugin-name.php gibt im oben gezeigten Beispielaufbau die zentrale Plugindatei an.
Diese Datei muss damit der Ordner als Plugin verstanden wird einen von Wordpress vorgegebene Headerstrutkur enthalten.
Das Minimum des Header ist wie folgt definiert:

\begin{lstlisting}[caption={Minimaler Plugin-Header in WordPress}, label={lst:plugin-header}]
<?php
/*
 * Plugin Name: YOUR PLUGIN NAME
 */
\end{lstlisting}

Darüber hinaus können weitere optionale Felder definiert werden wie:
\texttt{Author URI} für die Website des Plugin-Entwicklers, \texttt{Text Domain}
zur Internationalisierung und \texttt{Domain Path} für den Pfad zu Sprachdateien.
Alle möglichen Felder werden in der folgenden Tabelle aufgeführt:
\renewcommand{\arraystretch}{1.3}

\begin{longtable}{>{\bfseries}p{4cm} p{10cm}}
 \caption{Übersicht der Plugin-Header-Felder in WordPress} \\
 \hline
 Feldname & Beschreibung \\
 \hline
 \endfirsthead

 \multicolumn{2}{l}{\textit{Fortsetzung von vorheriger Seite}} \\
 \hline
 Feldname & Beschreibung \\
 \hline
 \endhead

 \hline
 \multicolumn{2}{r}{\textit{Fortsetzung auf nächster Seite}} \\
 \endfoot

 \hline
 \endlastfoot

 Plugin Name (erforderlich) & Der Name des Plugins, wie er in der Plugin-Übersicht im WordPress-Adminbereich angezeigt wird. \\
 Plugin URI & Eine eindeutige URL zur Startseite des Plugins, idealerweise auf der eigenen Website. WordPress.org-URLs sind hier nicht erlaubt. \\
 Description & Eine kurze Beschreibung des Plugins (max. 140 Zeichen), wie sie im WordPress-Adminbereich angezeigt wird. \\
 Version & Die aktuelle Versionsnummer des Plugins, z.\,B. \texttt{1.0} oder \texttt{1.0.3}. \\
 Requires at least & Die minimale WordPress-Version, mit der das Plugin kompatibel ist. \\
 Requires PHP & Die minimale benötigte PHP-Version für das Plugin. \\
 Author & Der Name des Autors oder der Autoren des Plugins. Mehrere Autoren können durch Kommas getrennt angegeben werden. \\
 Author URI & Die Website des Autors oder ein öffentliches Profil, z.\,B. auf WordPress.org. \\
 License & Die Kurzbezeichnung der Lizenz, unter der das Plugin veröffentlicht wird (z.\,B. \texttt{GPLv2}). \\
 License URI & Ein Link zum vollständigen Lizenztext (z.\,B. \url{https://www.gnu.org/licenses/gpl-2.0.html}). \\
 Text Domain & Die Textdomain für die Internationalisierung des Plugins, erforderlich für Übersetzungen mit \texttt{gettext}. \\
 Domain Path & Das Verzeichnis, in dem WordPress die Übersetzungsdateien findet (z.\,B. \texttt{/languages}). \\
 Network & Gibt an, ob das Plugin netzwerkweit (Multisite) aktiviert werden kann. Nur \texttt{true} ist erlaubt, andernfalls Feld weglassen. \\
 Update URI & Eine eindeutige URI zur Updatequelle, um versehentliche Überschreibungen durch gleichnamige Plugins im WordPress.org-Verzeichnis zu vermeiden. \\
 Requires Plugins & Eine durch Kommas getrennte Liste von Plugin-Slugs aus dem WordPress.org-Verzeichnis, die als Abhängigkeit benötigt werden. Kommas innerhalb der Slugs sind nicht erlaubt. \\
\end{longtable}

\newpage
\textbf{Hooks: Action und Filter}\\\\
Es gibt eine Kardinalregel in der Wordpress Entwicklung die besagt: Don’t touch WordPress core. %TODO: https://developer.wordpress.org/plugins/intro/ maybe verlinken?
Damit Anpassungen an bestehende Funktionalitäten möglich sind gibt es Hooks und Filter.
Hooks erlauben es an spezifischen Stellen einzugreifen, um das Verhalten von Wordpress zu ändern ohne dabei die Kern-Dateien bearbeiten zu müssen.
Allgemein gibt es zwei Arten von hooks in Wordpress: Actions und Filter.
Mit Actions können Funktionen hinzugefügt oder geändert werden.
Filter wiederum ändern Inhalte, während diese geladen und dem Website-Nutzer angezeigt werden.
Hooks sind nicht nur für die Plugin-Entwicklung gedacht, sondern werden ebenfalls häufig verwendet, um Standardfunktionen durch den Wordpress-Kern selbst bereitzustellen.

Drei sehr wichtige Hooks im Bereich der Plugins zählen der \texttt{register\_activation\_hook()}, \texttt{register\_deactivation\_hook()} und der \texttt{register\_uninstall\_hook()}.

\begin{itemize}
 \item register\_activation\_hook(): Dieser Hook wird ausgeführt, wenn das Plugin im Wordpress-Backend aktiviert wird. Dies wird oftmals verwendet um Funktionen zwecks Einrichtung des Plugins bereitzustellen.
 \item register\_deactivation\_hook(): Der Deaktivierung-Hook wird ausgeführt, wenn das Plugin deaktiviert wird. Diese Funktion ermöglicht es die vom Plugin temporär gespeicherten Daten und Einträge zu entfernen.
 \item register\_uninstall\_hook(): Die Ausführung hier findet statt, wenn das Plugin aus dem Backend gelöscht wird. Die Verwendung zielt häufig darauf ab alle vom Plugin erstellen Optionen oder Datenbanktabellen restlos zu löschen.
\end{itemize}

%Plugin-Struktur und -Organisation

%Das WordPress Hook-System verstehen

%Plugin-Header und Metadaten

%Aktivierung und Deaktivierung von Plugins

%Plugin-Hauptdatei und Ordnerstruktur

\subsection{Best Practises in der Plugin-Programmierung}\
Die Wordpress Developer Resources geben dem Entwickler einige Best Practises vor, welche helfen den Code zu organisieren.
Ferner wird durch die Hinweise sichergestellt, dass der Code gut mit dem Wordpress Kern und anderen Plugins funktioniert.
Nachfolgend sollen wichtige Best Practise aufgeführt werden, welche im Verlauf der Arbeit mit in der Konzeption berücksichtigt werden.
\\
\\
\textbf{Vermeiden von Namenskonflikten}\\\\
Namenskollisionen können durch die gleiche Benennung von Variablen, Funktionen oder Klassen zustandekommen.
Um einen solchen Namenskonflikt zu vermeiden ist es empfohlen einen globalen Namenspace zu definieren.
Dieser Namespace ermöglicht das überschreiben von anderen Variablen, Funktionen und Plugins.
Variablen die innerhalb von Funktionen oder Klassen definiert sind, sind hiervon nicht betroffen.
\\
\\
\\
\textbf{Voranstellen eines Prefix}\\\\
Unter Zuhilfenahme eines Prefix können potenzielle Konflikte im Hinblick auf Aufrufe und Ausführungen mit anderen Plugins verhindert werden.
Das Handbuch empfiehlt hierzu ein einzigartiges Wort, welches vor verschiedene deklarationen vorangestellt wird.
Auf das Projekt ChariGame gemünzte Projekt könnte dies dann wie folgt aussehen:
\begin{itemize}
 \item function charigame\_save\_post();
 \item define ('CHARIGAME\_LICENSE', true);
 \item class CHARIGAME\_Admin{}
 \item namespace ChariGame;
 \item update\_option('charigame\_settings', \$settings)
\end{itemize}
\vspace{1em}
\textbf{Prüfung vorhandener Implementierungen}\\\\
Damit der Code Robust  und weniger fehleranfällig ist, gilt es auf bestehende Implementationen zu prüfen.
Dies geschieht unter Zuhilfenahme der folgenden von PHP zur Verfügung gestellten Funktionen:
\begin{itemize}
 \item Variablen: isset()
 \item Function\_exists() %NICHT BENÖTIGT IN EIGENEM NAMESPACE
 \item Classes class\_exists()
 \item Constants defined()
\end{itemize}
\vspace{1em}
\textbf{Architekturmuster}\\\\
Die Wordpress Developer Resources geben eine Hand voll mögliche Architekturmuster vor.
Diese können grob in drei Variationen kategorisiert werden:
\begin{itemize}
 \item Einzelne Plugin-Datei, die Funktionen enthält
 \item Einzelne Plugin-Datei, die eine Klasse, ein instanziiertes Objekt und optional Funktionen enthält
 \item Haupt-Plugin-Datei, dann eine oder mehrere Klassendateien
\end{itemize}

Je nachdem welchen Umfang das Plugin anstrebt, gilt es die passende Lösung des Architekturmusters zu wählen.
Ein einzelne Plugin-Datei, die Funktionen enthält stellt eine solide Basis für ein kleines Plugin zur Verfügung.
Die Variante eine Haupt-Plugin-Datei zu erstellen und mehrere Klassendateien aufzubauen bietet sich eher für Plugins mit einer größeren Codebasis an.

\subsection{Plugin Security und Privacy}
Ein wichtiger Aspekt bei der Programmierung von Wordpress Plugins ist die Plugin Security und Privacy.
Damit die Sicherheit von Eingaben gegeben ist, werden Methoden wie Validating, Sanatizing und Escaping genutzt.

\textbf{Validating}

\begin{quote}
 Das Validierung von Eingaben ist der Prozess, bei dem Daten anhand eines vordefinierten Musters (oder mehrerer Muster) mit einem eindeutigen Ergebnis getestet werden: gültig oder ungültig.
 Die Validierung ist im Vergleich zur Bereinigung ein spezifischerer Ansatz, aber beide haben ihre Berechtigung.
 \\[0.5em]
 \emph{(eigene Übersetzung nach \cite{wordpress2024plugin_validation})}
\end{quote}

Es werden auch seitens der Wordpress Developer Ressourcen einfache Validierungsbeispiele genannt:
\begin{itemize}
\item Überprüfung, ob Pflichtfelder ausgefüllt wurden
\item Überprüfung, ob eine eingegebene Telefonnummer nur Zahlen und Satzzeichen enthält
\item Überprüfung, ob eine eingegebene Zeichenfolge eine von fünf gültigen Optionen ist
\item Überprüfung, ob ein Mengenfeld größer als 0 ist
\end{itemize}
Die Validierung selbst wird in verschiedene Philosophien gegliedert, die nun kurz exemplarisch aufgefasst werden:
\begin{table}[h]
 \centering
 \renewcommand{\arraystretch}{1.3}
 \begin{tabular}{|p{3cm}|p{5cm}|p{6cm}|}
  \hline
  \textbf{Philosophie} & \textbf{Beschreibung} & \textbf{Beispiel Anwendung} \\
  \hline
  Safelist \newline (Allowlist)
  & Nur Werte aus einer vordefinierten Liste erlauben. Alles andere wird abgelehnt.
  & Dropdown-Auswahl für Sortieroptionen (z.\,B. ``date``, ``author``). Nur erlaubte Keys werden akzeptiert. \\
  \hline
  Blocklist \newline (Denylist)
  & Bestimmte bekannte, unerwünschte Werte verbieten. Unsicher, da neue unerlaubte Werte nicht erfasst werden.
  & Sperren von bekannten schädlichen Dateiendungen (z.\,B. ``.exe``, ``.bat``). \\
  \hline
  Format Detection
  & Prüfen, ob Eingabe einem bestimmten Muster entspricht.
  & Postleitzahl-Validierung mit Regex, E-Mail-Prüfung mit \texttt{is\_email()}, nur Ziffern mit \texttt{ctype\_digit()}. \\
  \hline
  Format Correction (Sanitization)
  & Eingaben akzeptieren, aber in ein sicheres Format umwandeln.
  & Benutzername \newline mit \texttt{sanitize\_title()}, Typ-Cast von String zu Integer \texttt{(int)\$input}. \\
  \hline
 \end{tabular}
 \caption{Übersicht Validierungs-Philosophien mit Beispielen}
\end{table}

\newpage
Wenn keine Validierung möglich ist bedienen sich Wordpress Entwickler der Möglichkeit des Sanatizing und Escaping von Werten.

\textbf{Sanitizing}
\begin{quote}
 Die Bereinigung von Eingaben ist der Prozess der Sicherung/Bereinigung/Filterung von Eingabedaten.
 Die Validierung ist der Bereinigung vorzuziehen, da sie spezifischer ist.
 Wenn jedoch eine „spezifischere“ Lösung nicht möglich ist, ist die Bereinigung die nächstbeste Option.
 \\[0.5em]
 \emph{(eigene Übersetzung nach \cite{wordpress2024plugin_sanitizing})}
\end{quote}


Bevor also Daten, die von Nutzenden stammen, in der Datenbank abgelegt oder weiterverarbeitet werden,
sollten sie bereinigt werden, um sowohl Sicherheitslücken als auch ungewollte Inhalte zu vermeiden.
Dafür stellt WordPress verschiedene Funktionen bereit, die sicherstellen, dass Eingaben ausschließlich
die vorgesehenen Zeichen oder Strukturen enthalten.
Ein häufig genutztes Beispiel ist die Verwendung von \texttt{sanitize\_text\_field()} für Freitextfelder.
Nachfolgend werden die seitens Wordpress bereitgestellten Funktionen erläutert:

\begin{table}[h]
 \centering
 \begin{tabular}{|l|p{8cm}|}
  \hline
  \textbf{Funktion} & \textbf{Beschreibung} \\
  \hline
  \texttt{sanitize\_text\_field()} & Entfernt Tags, ungültige UTF-8-Zeichen, Zeilenumbrüche, Tabs und überflüssige Leerzeichen. \\
  \hline
  \texttt{sanitize\_email()} & Validiert und bereinigt eine E-Mail-Adresse. \\
  \hline
  \texttt{sanitize\_file\_name()} & Entfernt unerlaubte Zeichen aus Dateinamen. \\
  \hline
  \texttt{sanitize\_hex\_color()} & Prüft und bereinigt Hex-Farbwerte (\#RRGGBB). \\
  \hline
  \texttt{sanitize\_html\_class()} & Bereinigt Klassennamen für HTML-Attribute. \\
  \hline
  \texttt{sanitize\_key()} & Bereinigt Array-Keys oder Optionsnamen. \\
  \hline
  \texttt{sanitize\_option()} & Bereinigt Optionswerte in der Datenbank. \\
  \hline
  \texttt{sanitize\_title()} & Macht einen String zu einem URL-freundlichen Titel. \\
  \hline
  \texttt{sanitize\_user()} & Bereinigt Benutzernamen. \\
  \hline
  \texttt{wp\_kses()} & Filtert HTML nach erlaubten Tags/Attributen. \\
  \hline
  \texttt{wp\_kses\_post()} & Wie \texttt{wp\_kses()}, aber mit Standardregeln für Posts. \\
  \hline
 \end{tabular}
 \caption{Auswahl wichtiger WordPress-Sanitization-Funktionen}
\end{table}

\newpage

\textbf{Escaping}\\\\
Escaping transformiert Ausgabedaten so, dass gefährliche Zeichen (z.\,B. `<` wird zu `\&lt;`) nicht mehr als Code interpretiert werden. Dies verhindert XSS-Angriffe.\\\\
Das Prinzip lautet: Daten unverändert speichern, erst bei der Ausgabe escapieren. WordPress bietet kontextspezifische Funktionen wie \texttt{esc\_html()}, \texttt{esc\_attr()}, \texttt{esc\_url()} und \texttt{esc\_js()}.
Die richtige Funktionswahl je nach Ausgabekontext ist entscheidend für die Sicherheit.
Genau wie beim Sanatizing bietet Wordpress hier ein Set an Funktionen, welche speziell für das Escaping genutzt werden.

\begin{table}[h]
 \centering
 \begin{tabular}{|l|p{9cm}|}
  \hline
  \textbf{Funktion} & \textbf{Anwendungsfall / Beschreibung} \\
  \hline
  \texttt{esc\_html()} & Escaped Text für den Einsatz innerhalb von HTML-Elementen. Entfernt sämtliches HTML. \\
  \hline
  \texttt{esc\_attr()} & Für Werte innerhalb von HTML-Attributen, z.\,B. \texttt{<input value="...">}. \\
  \hline
  \texttt{esc\_url()} & Für URLs in \texttt{src} oder \texttt{href}-Attributen. \\
  \hline
  \texttt{esc\_url\_raw()} & Roh-URL-Escaping für Speicherung in der Datenbank oder nicht-encodierte Weitergabe. \\
  \hline
  \texttt{esc\_js()} & Für Inline-JavaScript oder Übergabe von Variablen in Skripte. \\
  \hline
  \texttt{esc\_xml()} & Absicherung von Ausgaben in XML-/XSL-Kontexten. \\
  \hline
  \texttt{esc\_textarea()} & Encodiert Texte für die Verwendung innerhalb eines \texttt{<textarea>} Elements. \\
  \hline
  \texttt{wp\_kses()} & Filtert HTML-Inhalte und erlaubt nur eine definierte Menge an Tags/Attributen. \\
  \hline
  \texttt{wp\_kses\_post()} & Variante von \texttt{wp\_kses()}, die die Standardmenge an Post-HTML erlaubt. \\
  \hline
  \texttt{wp\_kses\_data()} & Variante von \texttt{wp\_kses()}, die nur HTML erlaubt, das in Kommentaren zugelassen ist. \\
  \hline
 \end{tabular}
 \caption{Überblick zentraler Escaping-Funktionen in WordPress}
 \label{tab:escaping}
\end{table}

Ein wichtiger Grundsatz ist das \emph{späte Escaping}: Daten sollten erst unmittelbar vor der Ausgabe escaped werden.
Dadurch wird die Code-Überprüfung vereinfacht, das Fehlerrisiko verringert und die Anwendung widerstandsfähiger gegen zukünftige Anpassungen.
Wenn ein spätes Escaping nicht umsetzbar ist (etwa bei generiertem JavaScript-Code), muss der Entwickler sicherstellen, dass Variablen bereits in einer als \enquote{sicher} gekennzeichneten Form vorliegen (z.\,B. \texttt{\$variable\_escaped}).

\textbf{Nonces}

Ein Nonce (number used once) ist ein Sicherheitsmechanismus in WordPress zum Schutz von URLs und Formularen vor unbefugten Manipulationen.
Insbesondere wird der Schutz vor Cross-Site Request Forgery (CSRF) sichergestellt.
Anders als die Definition handelt es sich nicht um Einmalzahlen, sondern um Hash-Werte.
Diese Hashwerte sind für einen bestimmten Benutzer und Kontext über eine begrenzte Zeit gültig (Standard: 12–24 Stunden).

Erstellung: \texttt{wp\_create\_nonce('action')}, \texttt{wp\_nonce\_url()}, \texttt{wp\_nonce\_field()}

Validierung: \texttt{wp\_verify\_nonce()}, \texttt{check\_admin\_referer()}, \texttt{check\_ajax\_referer()}

Sie sollten stets aktionsspezifisch sein und niemals als alleinige Sicherheitsmaßnahme dienen.




%TODO: Mögliche erweiterung ist die wp template hiearchie? nochmal schön n bild rienfetzen hat ja was mit dem template von charigame zu tun FE ? maybe
\subsection{Weiterführende Funktionen und Paradigma}
In diesem Kapitel sind weitere für das Projekt verwendete Funktionen und Paradigmen aus Wordpress aufgeführt.
\\
\\
\textbf{Themes}\\\\
Ein Theme in WordPress bestimmt das Erscheinungsbild einer Website.
Es legt Farben, Schriftarten, Layout-Strukturen sowie die gesamte Darstellung des Frontends fest und kann auch Funktionen im Backend hinzufügen.
Für das Projekt Charigame spielt das konkret gewählte Theme jedoch keine zentrale Rolle.
Die entwickelte Lösung soll unabhängig von einem bestimmten Theme funktionieren.
%Die Definition dient hier daher lediglich dazu, ein grundlegendes Verständnis für den Aufbau und die Funktionsweise von WordPress zu vermitteln.



%Shortcodes
%Shortcodes sind im Grunde Makro, die eine komplexe Funktion aufrufen können.
\\
\\
\textbf{Metadaten}\\\\
Unter Metadaten versteht man ergänzende Informationen zu bestehenden Inhalten sogenannte „Informationen über Informationen".
Im WordPress-Kontext handelt es sich dabei um zusätzliche Datensätze, die verschiedenen Objekten wie Artikeln, Nutzerprofilen, Kommentaren oder Kategorien hinzugefügt werden können.
Praktische Anwendungsbeispiele sind etwa Custom Fields in Blogbeiträgen, erweiterte Benutzerinformationen in Profilen oder spezielle Attribute für Kategorienseiten.
WordPress zeichnet sich durch ein besonders anpassungsfähiges Metadaten-System aus.
Jedes Element kann mit einer unbegrenzten Anzahl zusätzlicher Informationen ausgestattet werden.
\\
\\
\textbf{API Endpoints}\\\\
Die WordPress-REST-API stellt unter dem Endpunkt /wp-json/ eine Schnittstelle zur Verfügung, über die Inhalte und Strukturen einer WordPress-Installation abgerufen und verändert werden können.
Der zentrale Namespace wp/v2/ umfasst dabei alle wesentlichen Ressourcen, wie Beiträge, Seiten, Benutzer, Medien oder Taxonomien.
\\
Technisch fundiert die Kommunikation im Gutenberg-Editor selbst auf der REST-API.
Änderungen im Block Editor werden im Hintergrund als Anfragen an die API umgesetzt.
Entwickler können über die JavaScript-Schnittstellen wie @wordpress/api-fetch eigene Blöcke implementieren oder externe Datenquellen einbinden.
\\
\\
\textbf{AJAX}\\\\
Wordpress bietet eine eigene AJAX-Schnittstelle, die auf den Endpunkt admin-ajax.php verweist.
Entwickler können hier eigene Aktionen registrieren, die bei einer AJAX-Anfrage ausgeführt werden.
Jede Aktion wird über einen eindeutigen Namen (action) identifiziert, wodurch mehrere unabhängige AJAX-Funktionen parallel existieren können.
Die Kommunikation erfolgt typischerweise über POST- oder GET-Anfragen, die Daten in JSON- oder URL-kodierter Form zurückliefern.
\\
\\
\textbf{WP-CRON}\\\\
In WordPress werden zeitgesteuerte Aufgaben über ein eigenes System, das sogenannte WP-Cron.
WP-Cron bietet eine einfache Möglichkeit, zeitbasierte Prozesse zu implementieren.
Solche Events werden typischerweise beim Planen von Beiträgen oder dem Versenden von E-Mail-Benachrichtigungen genutzt.

%Hier noch überlegen welche Funktionen und Paradigmen aufgeführt werden sollten. Effektiv bisschen filler für das kommende dann?
%Meta-Daten // Meta Boxes für Post-Bearbeitung

%Einstellungsseiten mit Settings \gls{api}

%Integration mit \gls{acf} für erweiterte Felder

%Shortcodes entwickeln und registrieren

%\gls{css} und JavaScript richtig einbinden

%Gutenberg Block-Entwicklung (optional)

%Custom \gls{rest} \gls{api} Endpoints erstellen

%Daten über \gls{http}-Requests bereitstellen

%Authentifizierung und Berechtigungen

%AJAX-Funktionalität implementieren

%Admin-Menüs und Unterseiten erstellen

%WordPress Database \gls{api} für Datenbankoperationen

%Custom Post Types und Custom Fields

%Settings und Options verwalten

%Caching mit Transients

%Plugin für WordPress Repository vorbereiten

%Versionskontrolle und Updates

%Lizenzierung unter \gls{gpl}

%Plugin-Testing und Qualitätssicherung

%Cron

\section{Gutenberg-Editor: Konzept und technische Grundlagen}
\begin{quote}
    Der Block-Editor ist ein modernes Paradigma für die Erstellung und Veröffentlichung von WordPress-Websites.
    Er verwendet ein modulares System aus Blöcken zum Erstellen und Formatieren von Inhalten und wurde entwickelt,
    um reichhaltige und flexible Layouts für Websites und digitale Produkte zu erstellen.
    \\[0.5em]
    \emph{(eigene Übersetzung nach \cite{wordpress2024plugin_blockeditor})}
\end{quote}



Ein Block ist ein eigenständiges Element wie beispielsweise ein Absatz, eine Überschrift, ein Medienelement oder eine Einbettung.
Jeder Block wird als separates Element mit individuellen Bearbeitungs- und Formatierungsoptionen behandelt.
Wenn alle diese Komponenten zusammengefügt werden, bilden sie den Inhalt der Seite oder des Beitrags, der dann in der WordPress-Datenbank gespeichert wird.
\\
\\
Die Relevanz des Editors wird durch die Erhobene Statistik der Seite Gutenberg in Numbers deutlich.
Demnach beträgt die Anzahl aktiver Installationen des Gutenberg-Editors auf WordPress- und Jetpack-Seiten aktuell 87,3 Mio.
Innerhalb der letzten drei Jahre wurden 282 Mio. Beiträge mit dem Gutenberg-Editor erstellt.
Die Daten basieren auf einem Dreijahreszeitraum und umfassen nur WordPress.com- und Jetpack-Seiten, die die Nutzung des Editors melden.\cite{gutenstats}

\subsection{Dateistruktur eines Blocks}
Die Developer Resources geben Entwicklern Grundlagen der Blockentwicklung mit, welche nachfolgend erläutert werden.

Seitens Wordpress wird eine Dateistrutkur für die Gutenberg Blöcke empfohlen.
Diese Struktur erleichtert die Trennung von Metadaten, Logik und der Darstellung.
Abbildung~\ref{fig:block-structure} zeigt den grundsätzlichen Aufbau eines Blocks, wie er in der offiziellen Dokumentation beschrieben ist.

\begin{figure}[h!]
    \centering
    \includegraphics[width=\textwidth]{block-structure.png} % füllt gesamte Textbreite
    \caption{Dateistruktur eines WordPress-Blocks \cite{wordpress2023}}
    \label{fig:block-structure}
\end{figure}

Die Abbildung verdeutlicht, dass zentrale Dateien wie \texttt{block.json} zur Definition der Block-Metadaten,
sowie Skripte, Stylesheets und Build-Artefakte in klar getrennten Verzeichnissen organisiert werden.


%\begin{figure}[tbh]
%    \centering
%    \includegraphics[width=0.88\textwidth]{gutenbergblock}
%    \caption{Gutenberg Block Struktur}
%    \label{fig:gutenberg-block-struktur}
%\end{figure}

Wenn ein Block entwickelt wird, ist es angeraten diesen als \gls{Plugin} und nicht innerhalb des Themes zu registrieren.
Dies hat den Vorteil, dass der erstellte Block unabhängig vom verwendeten Theme genutzt werden kann.
\newpage
\subsection{Schlüsselkonzepte im Block Editor}
\textbf{Kombinierbarkeit}
\\\\
Blöcke sind darauf ausgelegt, auf vielfältige Weise miteinander kombiniert zu werden.
Sie folgen einem hierarchischen Aufbau, bei dem ein Block in einen anderen eingebettet werden kann.
Verschachtelte Blöcke und ihr umschließender Block werden auch als Kinder und Eltern bezeichnet.
Ein Spalten-Block kann beispielsweise als übergeordneter Block fungieren und mehrere untergeordnete Blöcke in seinen einzelnen Spalten enthalten.
\\
Die Programmierschnittstelle, die die Verwendung von Unter-Blöcken regelt, wird InnerBlocks bezeichnet.


\\\\
\textbf{Daten und Attribute}
\\\\
Blöcke verstehen Inhalte als Attribute.
Ein Block kann eine beliebige Anzahl von Attributen beinhalten und diese werden in einer Key Value Relation definiert.
Der Key spiegelt den Namen wider und der Value ist der Wert.
Diese Inhalte sind serialisierbar in HTML und werden im Block HTML gespeichert und so auch in der Datenbank als post\_content abgelegt.


\begin{lstlisting}[caption={Beispiel eines Gutenberg-Blocks in WordPress}]
<!-- wp:paragraph {"key": "value"} -->
<p>Welcome to the world of blocks.</p>
<!-- /wp:paragraph -->
\end{lstlisting}

Blöcke lassen sich in zwei Kategorien unterteilen: statisch und dynamisch.
Statische Blöcke bestehen aus bereits gerenderten Inhalten sowie einem Attribut-Objekt, das bei Änderungen für das erneute Rendern verwendet wird.
Dynamische Blöcke hingegen benötigen serverseitige Daten und werden erst während der Inhaltsgenerierung gerendert.




\\\\
\textbf{Blockmuster}
\\\\
Ein Blockermuster oder auch Pattern ist eine Gruppe von Blöcken, die zu einem Designmuster zusammengefasst werden.
Solche Designmuster bieten für die schnellere Erstellung komplexer Seiten einen Ausgangspunkt.
Die Größe des Blockmusters ist Variabel und kann von einem einzelnen Block bis hin zu einer gesamten Seite reichen.
Themes können Designmuster registrieren, um Benutzern schnelle Startpunkte für das Verwenden von Blöcken zu bieten.

\\\\
\textbf{Block Themes}
\\\\
Anders als der Beitragseditor konzentrieren sich die Block Themes auf die Deklaration und das Bearbeiten der gesamten Website.
Hier können unter Zuhilfenahme von Blöcken die Kopf- bis hin zur Fußzeile Inhalte angepasst werden.
Die Block Themes selbst werden unterteilt in Vorlagen, die die gesamte Seite beschreiben oder Vorlagenteile, die wiederverwendbare Bereich sein können.
Angepasste Blockvorlagen umfassen statische als auch dynamische Seiten, wie Archive, Singular, Home, 404 usw. \cite{wordpress2024EditorTemplates}


\section{Gamification im Kontext digitaler Anwendungen}

Deterding et al. definieren Gamification als ``the use of game design elements in non-game contexts''.\cite{deterding2011gamification}
Diese Definition erfasst ein Phänomen, das in seiner Grundform nicht neu ist.
Leistungsbezogene Vergütungssysteme, Rankings und spielerische Wettkämpfe existieren in Fabriken, dem Vertrieb und Bildungseinrichtungen bereits seit Jahrzehnten.\cite{bpb2023gamification}
Die Digitalisierung hat seit etwa 2010 eine weitreichende Transformation des Gamification-Konzepts eingeleitet.
Durch die Verbreitung von Smartphones und insbesondere Wearables wie Smartwatches und Fitnessbändern haben neue Dimensionen der spielerischen Motivationsgestaltung eröffnet.\cite{sailer2016gamification}
Diese Entwicklung stellt einen Wechsel in der Gestaltung des Mensch-Computer-Interaktionen Paradigma dar.

Gamification hat sich von simplen Belohnungssystemen zu einem komplexen Gesamtkonzept entwickelt, das die Wahl spezifischer Elemente und Mechaniken, die Berücksichtigung des Anwendungskontextes, der Zielgruppe sowie der zu erreichenden Ziele umfasst.\cite{bpb2023gamification}
\\\\
\textbf{Abgrenzung von Serious Games}\\\\
Im Projektkontext von Charigame ist eine klare Abgrenzung zwischen Gamification und Serious Games erforderlich.
Während Serious Games als eigenständige Spiele konzipiert werden, die primär pädagogische oder Lernziele verfolgen, versteht sich die Gamification-Implementierung in Charigame als funktionale Ergänzung.
Ziel des gamifizierten Parts ist es, die Nutzerbeteiligung zu steigern und nicht als Wissensvermittlung zu dienen.
\\
Charigame verbindet somit die konventionelle Spendenaktionen durch die Integration spielmechanischer Elemente, ohne dabei den ursprünglichen Zweck der Aktion zu verändern.
Der User Experience-orientierten Ansatz, zielt darauf ab, die Customer Journey zu optimieren und die emotionale Bindung der Nutzer zur Spendenaktion zu erhöhen.
Dieser spielerische Ansatz dient als Mittel zur Aktivierung und nicht als Selbstzweck.
\newpage
\textbf{Strategische Bedeutung von Gamification für Spendenaktionen}
\\\\
Die Integration von Gamification in Spendenaktionen adressiert mehrere strukturelle Herausforderungen traditioneller Fundraising-Ansätze.
Spendenaufrufe leiden häufig unter geringer Nutzerengagement und mangelnder Transparenz bezüglich der Verwendung der Mittel.
Gamification-Elemente schaffen hier Abhilfe durch die Visualisierung des sozialen Impacts und erhöhte Nutzerbeteiligung\cite{golrang2021applying}.
\\\\
Die Integration spielerischer Elemente steigert sowohl die Aufmerksamkeit als auch die Reichweite sozialer Kampagnen.
Parallel dazu ermöglicht dieser Ansatz eine authentische Kommunikation des Markenimages.
Das Unternehmen kann dadurch ihre gesellschaftliche Verantwortung auf eine partizipative Weise demonstrieren.
%TODO: Ziele mit einbeziehen //// Vielleicht jetzt auch nciht mher? genug text dunno
%8.3.1 Ziele
%Spielziele beschreiben konkrete, erreichbare Zustände und werden von allen
%Experten als wichtig betrachtet. Löwe findet gar, dass jedes User-Interface Ziele haben muss.
%Granaß stimmt zu, was die Wichtigkeit angeht, betont aber, dass das Ziel erreichbar sein
%muss: „[Das Ziel] darf aber auch nicht zu weit entfernt sein“ (GRANAß Z.109)
\section{Überblick über Coperate Responsibility und Charity-Plattformen}
Corporate Social Responsibility (CSR) hat sich in den vergangenen Jahrzehnten von einem optionalen Unternehmensengagement zu einem strategischen Kernbestandteil moderner Unternehmenspolitik entwickelt.
Die Europäische Kommission definiert CSR als die Verantwortung von Unternehmen für ihre Auswirkungen auf die Gesellschaft und beschreibt damit ein Konzept, das weit über die bloße Einhaltung gesetzlicher Bestimmungen hinausgeht.\cite{european_commission2011csr}
Diese gesellschaftliche Erwartungshaltung spiegelt sich in der zunehmenden Nachfrage nach transparenten und messbaren CSR-Maßnahmen wider, die sowohl von Stakeholdern als auch von Verbrauchern eingefordert werden.

Parallel zu dieser Entwicklung hat die Digitalisierung neue Möglichkeiten für die Umsetzung und Kommunikation von CSR-Initiativen eröffnet.
Online-Charity-Plattformen ermöglichen es Unternehmen heute, ihre gesellschaftliche Verantwortung auf innovative Weise wahrzunehmen und dabei gleichzeitig die Reichweite und Transparenz ihrer Aktivitäten zu erhöhen.
Diese digitalen Plattformen fungieren als Vermittler zwischen Unternehmen, gemeinnützigen Organisationen und der Öffentlichkeit und schaffen neue Formen der partizipativen Philanthropie.


Marktanalyse für Charigame
Marktumfeld und Trends
Gamification-Markt in Deutschland
Der Gamification-Markt in Deutschland zeigt eine positive Entwicklung mit signifikantem Wachstum.
Unternehmen aus verschiedenen Branchen wie Gesundheitswesen, Finanzdienstleistungen und E-Learning setzen zunehmend auf gamifizierte Lösungen, um die Kundeninteraktion und Mitarbeiterbindung zu verbessern.

Europa verzeichnet insgesamt ein starkes Wachstum, das durch Unternehmensinvestitionen in Gamification-Konferenzen und KI-gestützte Engagement-Lösungen angetrieben wird.
Länder wie Deutschland, Großbritannien und Frankreich sind Vorreiter bei der Gamification in Einzelhandel, HR und Gesundheitswesen.

CSR-Trends 2025
Aktuelle Trends im Bereich Corporate Social Responsibility zeigen eine verstärkte Fokussierung auf Mitarbeiterengagement und direkte Beteiligung.
Unternehmen suchen nach innovativen Wegen, ihre CSR-Maßnahmen authentischer und partizipativer zu gestalten.

Competitive Landscape
Direkte Konkurrenz
Im WordPress-Ökosystem existieren bereits etablierte Spenden-Plugins wie:

Charitable: Das führende WordPress-Plugin für Spenden und Fundraising mit über 10.000 Non-Profit-Organisationen als Nutzer

GiveWP: Ein populäres Plugin mit umfangreichen Funktionen für Online-Spenden, Preise ab \$149 pro Jahr

Gamification-Anbieter in Deutschland
Der deutsche Markt umfasst verschiedene Gamification-Spezialisten:

Gamewheel (Berlin): One-Stop-Shop für Gamification und Playable Ads

Retrac GmbH (Berlin): White-Label-Gamification-Lösungen

Cluehub (Berlin): Spezialist für Serious Games und Team-Events

Alleinstellungsmerkmale von Charigame
Einzigartige Positionierung
Charigame besetzt eine Nischenpositioning an der Schnittstelle von:

Gamification

Corporate Social Responsibility

WordPress-Integration

Diese Kombination ist im deutschen Markt noch nicht etabliert. Bestehende Lösungen fokussieren entweder auf reine Spendenfunktionalität oder allgemeine Gamification, aber nicht auf die spielerische Gestaltung von CSR-Kampagnen.

Zielmarkt und Potenzial
Primäre Zielgruppe
Mittelständische Unternehmen und Konzerne mit:

Etablierten CSR-Programmen

WordPress-basierten Websites

Interesse an innovativen Kundenengagement-Strategien

Marktpotenzial
Der Markt zeigt positive Indikatoren:

Wachsende Nachfrage nach personalisierten und inklusiven Gamification-Designs

Unternehmen suchen nahtlos integrierte Erlebnisse, die über oberflächliche Anreize hinausgehen

Trend zu Micro-Moments und Hyper-Personalisierung in der Kundenbindung

Marktchancen und Herausforderungen
Chancen
Zeitgeist: Gamification wird zunehmend zur Förderung von Nachhaltigkeitsinitiativen eingesetzt, was perfekt zu Charigames CSR-Fokus passt.

Technologietrends: KI-gesteuerte Gamification-Designs und personalisierte Benutzererfahrungen werden 2025 dominierend.

Regulatorisches Umfeld: Deutsche DSGVO-Compliance kann als Vertrauensfaktor gegenüber internationalen Konkurrenten genutzt werden.

Herausforderungen
Marktreife: Kunden sind anspruchsvoller geworden und erwarten zielgerichtete, inklusive Gamification-Designs.

Preisdruck: Viele Kunden suchen "Luxus-Erfahrungen mit kleinem Budget".

Etablierte Konkurrenz: Bestehende Spenden-Plugins haben bereits große Nutzerbasen und umfangreiche Feature-Sets.

Markteintrittsempfehlungen
Positionierungsstrategie
Charigame sollte sich als "erste gamifizierte CSR-Lösung für WordPress" positionieren und den Mehrwert der spielerischen Spendererfahrung betonen.

Zielgruppenansprache
Fokus auf mittelständische Unternehmen, die bereits WordPress nutzen und ihre CSR-Aktivitäten digitalisieren möchten.

Competitive Advantages
Deutsche Entwicklung mit DSGVO-Compliance

Spezialisierung auf CSR-Gamification

WordPress-native Integration durch elancer-team's Expertise

Maßgeschneiderte Lösungen statt Standardprodukte

Der Markt für Charigame zeigt vielversprechende Perspektiven, da er eine bisher unbesetzte Nische zwischen Gamification und CSR-Management adressiert, während gleichzeitig von den positiven Trends in beiden Bereichen profitiert.


\label{chap:formal}
%
In diesem Kapitel finden Sie grundlegende Hinweise zum formalen Aufbau Ihrer Arbeit.
%
\textbf{Reihenfolge}
\label{sec:aufbau}
Eine wissenschaftliche Arbeit besteht in der Regel aus den folgenden Teilen:
%
\begin{enumerate}
 \item Deckblatt
 \item Kurzfassung/Abstract (optional)
 \item Inhaltsverzeichnis
 \item Abbildungs- und Tabellenverzeichnis (auch am Ende üblich)
 \item Abkürzungsverzeichnis (auch am Ende üblich)
 \item Einleitung
 \item Hauptteil
 \item Zusammenfassung/Fazit
 \item Literaturverzeichnis
 \item Anhänge (optional)
 \item Erklärung
\end{enumerate}
%
%
\textbf{Deckblatt}
%Die Gestaltung des Deckblatts folgt den visuellen Vorgaben für Publikationen der TH Köln.
%\par
Das Deckblatt beinhaltet: Titel der Arbeit, Art der Arbeit, Verfasser*in, Matrikelnummer, Abgabetermin, Betreuer*in sowie Zweitgutachter*in. Das Deckblatt wird bei Arbeiten, die länger sind als~15 Seiten, bei der Seitenanzahl zwar mitgezählt, jedoch nicht nummeriert.
%
%
\textbf{Inhaltsverzeichnis}
\label{sec:listOfContents}
Wir empfehlen eine Dezimalgliederung wie in diesem Dokument angelegt. Werden innerhalb eines Kapitels Unterüberschriften verwendet, müssen mindestens zwei vorhanden sein: wo ein~2.1 ist, muss es ein~2.2 geben.
\par
Das Inhaltsverzeichnis enthält immer die Seitenangaben zu den aufgelisteten Gliederungspunkten; es wird dabei aber selbst nicht im Inhaltsverzeichnis aufgelistet. Die Seiten, die das Inhaltsverzeichnis selbst einnimmt, können römisch gezählt werden.
%Mehr hierzu in Abschnitt~\cref{}.
\par
Für eine Abschlussarbeit ist eine Gliederungstiefe von wenigstens drei Ebenen üblich. In der Regel werden nur bis zu vier Ebenen vorne im Inhaltsverzeichnis abgebildet. Hier sollten Sie aber unbedingt die Gepflogenheiten in Ihrem Fach berücksichtigen und ggf. in Erfahrung bringen.
%\par
%In dieser Word-Vorlage wird das Inhaltsverzeichnis für die Überschriftenebenen 1 bis 3 automatisch generiert (Rechtsklick auf das Inhaltsverzeichnis > Felder aktualisieren > Ganzes Verzeichnis).
%
%
\textbf{Abbildungsverzeichnis und Tabellenverzeichnis}
Abbildungen und Tabellen werden in entsprechenden Verzeichnissen gelistet. In dieser Vorlage erscheinen sie direkt nach dem Inhaltsverzeichnis. Dann können die entsprechenden Seiten römisch gezählt werden. Die Verzeichnisse können jedoch auch am Ende der Arbeit vor oder hinter dem Literaturverzeichnis stehen. Dann werden sie regulär mit Seitenzahlen versehen.
%Verzeichnisüberschriften (z. B. Abbildungsverzeichnis) werden nie nummeriert (Formatvorlage Überschrift 1 unnummeriert verwenden).
%
%
\textbf{Abkürzungsverzeichnis}
Die Zahl der Abkürzungen sollte übersichtlich bleiben. Das Abkürzungsverzeichnis enthält lediglich wichtige fachspezifischen Abkürzungen in alphabetischer Reihenfolge, insbesondere Abkürzungen von Organisationen, Verbänden oder Gesetzen. Gängige Abkürzungen wie \enquote{u.\,a.}, \enquote{z.\,B.}, \enquote{etc.} werden nicht aufgenommen.
\par
Zur technischen Umsetzung mit \LaTeX{} vergleiche auch Abschnitt~\ref{sec:template}.
%
%
\textbf{Literaturverzeichnis}
Das Literaturverzeichnis wird alphabetisch nach Autorennamen geordnet. Es enthält alle im Text zitierten Quellen~--~und nur diese. Mehrere Schriften einer Person werden nach Erscheinungsjahr geordnet. Schriften derselben Person aus einem Erscheinungsjahr müssen Sie selbst unterscheidbar machen. In den Ingenieurwissenschaften wird zusätzlich häufig ein Nummern- oder Autorenkürzel dem Namen in eckigen Klammern voran-gestellt. Mehr hierzu und weitere wichtige Regeln des Zitierens lernen Sie in den E-Learning-Kursen des Schreibzentrums\footnote{\href{https://ilu.th-koeln.de/goto.php?target=cat\_52109\&client\_id=thkilu}{https://ilu.th-koeln.de/goto.php?target=cat\_52109\&client\_id=thkilu}} kennen.
\par
Zur Verwaltung der verwendeten Literatur eigenen sich entsprechende Softwaretools wie Citavi oder Zotero, die mit verschiedenen Textverarbeitungsprogrammen kompatibel sind.
%
%
\textbf{Rechtschreibung, Grammatik}
Achten Sie bei der Abgabe Ihrer Arbeit auf ein einwandfreies Deutsch bzw. Englisch. Wenn Fehler die Lesbarkeit beeinträchtigen, kann sich dies durchaus negativ auf die Note auswirken. Nutzen Sie daher unbedingt die Rechtschreibprüfung Ihres Textverarbeitungsprogramms, auch wenn diese nicht alle Fehler erkennt.
%In Word können Sie diese unter Datei > Optionen > Dokumentenprüfung bearbeiten sowie ein- und ausschalten.
\par
Für alle, die sich bei diesem Thema unsicher fühlen, empfehlen wir die E-Learning-Kurse des Schreibzentrums\footnote{\href{https://ilu.th-koeln.de/goto.php?target=cat\_52109\&client\_id=thkilu}{https://ilu.th-koeln.de/goto.php?target=cat\_52109\&client\_id=thkilu}}. Wenden Sie sich ggf. auch an die Beauftragte für Studierende mit Beeinträchtigung\footnote{\href{https://www.th-koeln.de/studium/studieren-mit-beeintraechtigung\_169.php}{https://www.th-koeln.de/studium/studieren-mit-beeintraechtigung\_169.php}}.
%
%
\textbf{Umfang der Arbeit}
Alle Fächer nennen verbindliche Angaben zu Unter- und Obergrenzen, die in der Regel eingehalten werden müssen. Verzeichnisse und Anhänge werden dabei in aller Regel nicht mitgezählt. In Einzelfällen~--~insbesondere bei empirischen Arbeiten~--~können abweichende Vereinbarungen mit der Betreuungsperson getroffen werden.
\chapter{Projektkontext}

\section{Das Projekt Charigame}

Das Projekt Charigame entstand ursprünglich als interne Initiative der elancer-team GmbH mit dem Ziel eine digitale Spendenaktion zu ermöglichen.
Diese Idee des Projekts lässt sich auf eine durch die Agentur erstellte Weihnachtsaktion zurückführen, die auf der Website \texttt{https://hohoho.elancer-team.de/} implementiert wurde.
Bei der Spendenaktion hat die elancer-team GmbH die Spendensumme bereitgestellt und die Agenturkunden wurden eingeladen teilzunehmen und durch ihr engagement den Spendentopf zu erhöhen.
Diese Plattform ermöglichte es den Kunden der Agentur erstmals, über die gamifizierte Benutzeroberfläche Spenden zu erspielen und zu verteilen.
\\\\
Die positive Resonanz der Agenturkunden auf diesen gamifizierten Ansatz führte zu der Überlegung das Konzept weiterzuentwickeln.
Aufgrund der erfolgreichen Durchführung äußerte ein Kunde der Agentur den Wunsch, eine vergleichbare Spendenaktion für das eigene Unternehmen zu realisieren.

Diese erste kundenspezifische Umsetzung basierte auf einer abgewandelten Kopie der Weihnachts-Spendenaktion.
Die Anfrage stellte dann den Übergang von einem internen Projekt zu einem eigenständigen Dienstleistungsangebot dar.
Daraufhin wurde durch den Autor das Projekt Charigame als eigenständiges Wordpress-Plugin realisiert.
Das daraus resultierende Wordpress-Plugin befindet sich aktuell bei einem Kunden im produktiven Einsatz und bildet die Grundlage für die in dieser Arbeit dokumentierten technischen Analyse und Weiterentwicklung.
\\\\
\textbf{Funktionsweise von Charigame}
\\\\
Die Funktionsweise von Charigame lässt sich grob in 5 Schritte unterteilen, die in Abbildung~\ref{fig:charigame-funktion}: veranschaulicht sind:

%\begin{figure}[!ht]
%    \centerline{\includesvg[width=1\columnwidth]{images/firstPartyCookie.svg}}
%    \caption{Entstehungsprozess von First Party Cookies}
%\end{figure}
\begin{figure}[tbh]
    \centering
    \includesvg[width=0.88\textwidth]{images/funktionsweise_charigame}
    \caption{Funktionsweise Charigame}
    \label{fig:charigame-funktion}
\end{figure}
\newpage

\begin{enumerate}
    \item \textbf{Aufbau der Kampagne und Mailing}
    \\ Eine Spendenkampagne wird im Backend von Charigame angelegt.
    Anschließend werden Personendaten der Kunden in das System importiert.
    Das System versendet dann, basierend auf den Kampagneneinstellungen, automatisierte E-Mails mit dem Link zur Spendenaktion.
    \item \textbf{Kundeninteraktion und Spendenspiel}
    \\ Der Kunde öffnet den personalisierten Code in der E-Mail oder gibt den darin enthaltenen Code auf der Login-Page ein.
    Die in der Kampagne eingestellte Charigame-Landingpage wird angezeigt und der Kunde kann aktiv beim Spendenspiel teilnehmen.
    \item \textbf{Spielende und Spendenverteilung}
    \\ Nachdem das Spendenspiel seitens des Kunden absolviert wurde, kann dieser den erspielten Beitrag prozentual auf einen von bis zu drei verschiedenen Spendenempfängern verteilen.
    \item \textbf{Dankesseite}
    \\ Eine Dankesseite wird angezeigt und der Kunde kann nach Belieben erneut an dem Spiel teilnehmen, um seine Punktzahl zu verbessern.
    Ferner werden weitere Handlungsauforderungen in Form von CTAs ausgespielt, die den Kunden gezielt auf Bereiche des Unternehmens leiten können.
\end{enumerate}

\section{Deskriptiver Stand}
\subsection{Darstellung und Einstellungen im Wordpress Front- und Backend}
Der deskriptive Stand von Charigame befindet sich in einem funktionsfähigen, jedoch technisch und konzeptionell ausbaufähigen Zustand.
Das Wordpress-Plugin integriert sich in das CMS Wordpress und erweitert den Funktionsumfang, um gamifizierte Spendenaktionen.
Bevor die Spendenaktion bereitsteht ist es notwendig, dass verschiedene Einstellungen getroffen.
Die notwendigen Einstellungen können über das Wordpress-Backend Menü aufgerufen werden.
\\
Hierzu erzeugt ein Charigame Menüpunkt im Wordpress Dashboard wie in Abbildung~\ref{fig:charigame-menu-legacy}: visualisiert.
In diesem Menü sind die wichtigsten Einstellmöglichkeiten als Menüpunkte definiert.
\begin{figure}[tbh]
    \centering
    \includegraphics[width=0.2\textwidth]{images/legacy_charigame_wordpress_menu}
    \caption{Charigame Menü im Wordpress Dashboard}
    \label{fig:charigame-menu-legacy}
\end{figure}

Die im Menü angezeigten Einstellungen und Informationen lassen sich in mehrere Kategorien gliedern.
Diese spiegeln den Aufbau der einzelnen Bausteine vom Projekt wider und werden nachfolgend erläutert.
\\\\
\textbf{General Settings}\\\\
Die General Settings enthalten grundlegende Angaben zum Unternehmen, das eine Spendenkampagne durchführt.
Neben rechtlichen Informationen wie dem Impressum und der AGBs können hier auch gestalterische Parameter wie Farben der Corporate Identity hinterlegt werden.
Die Backendansicht der Einstellungsseite ist in Abbildung~\ref{fig:charigame-general-settings-legacy} dargestellt.
\begin{figure}[H]
    \centering
    \includegraphics[width=0.7\textwidth]{images/legacy_general_settings}
    \caption{General Settings im WordPress Backend}
    \label{fig:charigame-general-settings-legacy}
\end{figure}

Im Detail können grundlegende Einstellungen konfiguriert werden, z.B.:
\begin{itemize}
\item Überschriften der Login-Form
\item Hintergrund- und Logo-Uploads
\item Farbschema (Primär, Sekundär, Teritär)
\end{itemize}
Eine vollständige Übersicht der Eingabefelder ist im Anhang zu finden (vgl. Tabelle~\ref{tab:eingabefelder_general_settings}).
\\\\
Die getroffenen Einstellungen bestimmen direkt das Erscheinungsbild der Login-Form im Frontend (vgl. Abbildung~\ref{fig:login-textkampagne}).

\begin{figure}[H]
    \centering
    \includegraphics[width=0.8\textwidth]{images/legacy_login_testkampagne}
    \caption{Login-Form einer Charigame-Testkampagne im Frontend}
    \label{fig:login-textkampagne}
\end{figure}

\textbf{Donation Recipients}\\\\
Die Donation Recipients stellen die Empfänger der Spenden dar.
Für jede Kampagne müssen mindestens drei Empfänger angelegt werden.
Pro Empfänger werden ein Name, ein Bild sowie ein beschreibender Text hinterlegt (vgl. Tabelle~\ref{tab:eingabefelder_donation_recipients}).
Das vorab erstellen der Recipients ist im Verlauf der Einstellungen zwingend erforderlich, da sie in den Kampagneneinstellungen referenziert werden.
Die Backendansicht der Donation Recipients ist in Abbildung~\ref{fig:donation-recipients-settings-legacy} veranschaulicht.
\begin{figure}[H]
    \centering
    \includegraphics[width=1\textwidth]{images/legacy_donation_recipients_backend}
    \caption{Donation Recipients im WordPress Backend}
    \label{fig:donation-recipients-settings-legacy}
\end{figure}
Die im Backend gesetzten Einstellungen werden in folgender Form wie (vgl. Abbildung~\ref{fig:donation-recipients-frontend-legacy}) im Frontend dargestellt:
\begin{figure}[H]
    \centering
    \includegraphics[width=1\textwidth]{images/legacy_donation_recipients_frontend}
    \caption{Donation Recipients im WordPress Frontend}
    \label{fig:donation-recipients-frontend-legacy}
\end{figure}

\newpage
\textbf{Game Types}\\\\
Unter Game Types werden die verfügbaren Spieltypen definiert.
Jedes Spiel kann vom Unternehmen mit einer \glqq How-To-Play\grqq{}-Anleitung versehen werden.
Diese besteht aus einer Abfolge von Schritten, die jeweils mit einem Icon, einer Überschrift und einem erklärenden Text ergänzt werden können.

Bild Backendansicht -- Bild Frontendansicht

\\\\
\textbf{Users \& Pipedrive Integration}\\\\
Im Menüpunkt Users können die Teilnehmer der Spendenkampagnen verwaltet werden.
Für ein Kundenprojekt wurde zudem eine Pipedrive-Integration entwickelt, welche über die \gls{REST}-\gls{API} automatisch Kontaktdaten aus dem CRM-System importiert.
\begin{itemize}
    \item Vorname
    \item Nachname
    \item Geburtsdatum
    \item E-Mail-Adresse
    \item Flags: Imported, E-Mail sent
\end{itemize}

Bild Backendansicht


\\\\
\textbf{E-Mail-Settings}\\\\
Das Plugin erlaubt es, anstelle des standardmäßigen WordPress-Mailers einen eigenen SMTP-Server einzubinden.
Dafür stehen folgende Eingabefelder bereit:
\begin{itemize}
    \item SMTP-Host
    \item SMTP-Port
    \item Benutzername
    \item Absender-E-Mail
    \item Absender-Name
\end{itemize}
Aus Sicherheitsgründen kann das Passwort optional in der wp-config.php als Konstante hinterlegt werden, sodass es nicht in der Datenbank gespeichert wird.
Zusätzlich bietet die Oberfläche eine Funktion zum Versand einer Testmail, mit der sich die Konfiguration überprüfen lässt.

Darüber hinaus bietet der Backend-View der E-Mail-Settings eine kleine Funktion, eine Testmail an eine ausgewählte Adresse zu senden um die Einstellungen auf Ihre richtigkeit prüfen zu können.

\\\\
\textbf{Data Table}\\\\
Die Data Table dient als zentrales Dashboard, um die Ergebnisse und Kennzahlen der Spendenkampagnen auf einen Blick darzustellen.
Angezeigt werden u. a.:
\begin{itemize}
    \item erstellte Kampagnen
    \item zugehörige Spendenempfänger
    \item Benutzername
    \item aktueller Spendenstand (in Euro)
    \item Nutzerübersicht mit einzigartigen Game Codes
\end{itemize}
Darüber hinaus werden weitere Metriken bereitgestellt, darunter Valid From, Valid Until, Last Played, Highscore, Recipient 1–3 sowie E-Mail sent.
Die genaue Bedeutung dieser Kennzahlen ist in Tabelle XY im Anhang erläutert.

%TODO: Umschreiben! Campaings einmal komplett und die oberen Punkte ebenfalls
\\\\
\textbf{Campaigns}\\\\
\\\\\\
Die Campaigns stellen das zentrale Element von Charigame dar und repräsentieren den konzeptionell und technisch komplexesten Bereich des Plugins. Dieses Modul führt sämtliche relevanten Informationen einer Spendenkampagne zusammen und ermöglicht deren Steuerung über umfangreiche Konfigurationsmöglichkeiten. Der gesamte Ablauf einer Kampagne kann somit innerhalb des WordPress-Backends abgebildet werden. Dies umfasst sowohl die Auswahl des Spielmechanismus als auch die Kommunikation mit den Teilnehmenden.

Zu den wesentlichen Komponenten gehören:

\begin{itemize}
\item \textbf{Allgemeine Kampagneninformationen} \
Jede Kampagne erhält einen Titel, eine kurze Bezeichnung sowie eine ausführliche Beschreibung. Diese Inhalte bilden die Grundlage für die Frontend-Darstellung und dienen der inhaltlichen Rahmung der Aktion. Ergänzende Medien wie Logos oder Illustrationen können hochgeladen und der Kampagne zugeordnet werden.
\item \textbf{Spielmechanik} \\
Der gewünschte Spieltyp wird aus den im System verfügbaren \textit{Game Types} ausgewählt. Dadurch wird die konkrete Mechanik der Spendenaktion definiert. Begleitende Hinweise, beispielsweise eine How-to-Play-Anleitung, können zusätzlich hinterlegt werden.

\item \textbf{Spendenempfänger (Recipients)} \\
Jede Kampagne bindet mindestens drei zuvor definierte \textit{Donation Recipients} ein. Diese Empfänger werden im Spiel zur Auswahl gestellt und gewährleisten die ordnungsgemäße Verteilung der generierten Spenden auf die vorgesehenen Organisationen.

\item \textbf{Punkte- und Spendenlogik} \\
Die Verknüpfung von Spielergebnissen mit der Spendenhöhe bildet ein zentrales Element. Über eine Highscore-Logik wird festgelegt, wie viele Punkte maximal erreichbar sind und welche Spendenwerte daraus resultieren. Gewinnkategorien ermöglichen eine Staffelung nach erreichten Punktzahlen (beispielsweise ab 20 Punkten 4 Euro, ab 50 Punkten 10 Euro). Dadurch entsteht ein direkter Zusammenhang zwischen Spielerfolg und Spendenbetrag. Ein übergeordnetes Spendenziel in Euro kann den Gesamtumfang der Kampagne begrenzen und als Fortschrittsindikator fungieren.

\item \textbf{Zeitliche Steuerung} \\
Kampagnen können zeitlich präzise geplant werden. Start- und Enddatum bestimmen den Gültigkeitszeitraum. E-Mails lassen sich automatisiert zu definierten Terminen versenden, um Teilnehmer gezielt zu aktivieren oder zu informieren.

\item \textbf{Kommunikation (E-Mail und Social Media)} \\
Für jede Kampagne können individuelle E-Mail-Texte konfiguriert werden. Diese umfassen Betreffzeile, Header, Haupttext sowie Signatur. Die Nachrichten informieren Nutzer über ihre Teilnahme, bedanken sich oder liefern Hintergrundinformationen zur Aktion. Optional kann ein E-Mail-Template mit HTML-Header und Footer gestaltet werden. Social-Media-Kanäle können durch die Hinterlegung von Links und Icons eingebunden werden, um die Reichweite der Kampagne zu erhöhen.

\item \textbf{Call-to-Action (CTA)} \\
Eine gezielte Weiterleitung kann eingerichtet werden. Hierzu werden Headline, Beschreibungstext, Button-Beschriftung und Ziel-URL definiert. Der CTA motiviert Teilnehmende nach Abschluss des Spiels oder nach Erhalt einer E-Mail zu weiteren Aktionen, beispielsweise dem Besuch einer Unternehmensseite oder einer weiterführenden Spendenkampagne.
\end{itemize}

Das Campaign-Modul ermöglicht nicht nur die inhaltliche und technische Konfiguration einzelner Kampagnen, sondern vereint auch die Schnittstellen zu Spendenlogik, Kommunikation und externer Reichweitensteigerung. Es bildet somit die Schaltzentrale von Charigame, in der organisatorische, spielmechanische und kommunikative Elemente in einer konsistenten Benutzeroberfläche zusammengeführt werden.

GSAP als Animation
Confetti JS Animation
NPM Pakage Manager
Tailwind CSS

Anhang
Login Page --
Gesamte Landingapge --
Dankesseite --

\subsection{Architektur}
So und dann auf Architektur eingehen.
CPTs nice
nur eine zentrale Plugin Datei
Screenshots Frontend Backend zeigen
Abhängigkeit ACF Pro generell acf als weiteres plugin

Bearbeitung von LP nur sehr eingeschränkt möglich viele texte statisch hinterlegt
Dashboard sehr Rudimentär und keine guten KPIs oder einblicke gesxhweige denn gutes visual design sehr einfach
Security keinen tiefergehenden augenmerk ?? Das vllt sogar außen vor lassen man munkelt....
https://natuerlich.reisen/wp-admin/post.php?post=999998&action=edit -- code ist nicht einzigartig
\section{Stakeholder und Nutzungskontext}
Als Stakeholder wurden seitens der elancer-team GmbH die folgenden Gruppen identifiziert:
Viereck mit den Stakeholdern.

Nutzungskontext: Für

\section{Gesetzte Anforderungen an die Weiterentwicklung}

Für den zu bearbeitenden Part im Kaptiel 4 sind Praxispart mindestens die folgenden Anforderungen gesetzt:
\subsection{Technische Anforderungen}
\begin{itemize}
    \item Das System muss die Abhängigkeit von der PRO-Version des Plugins ACF (Advanced Custom Fields) entfernen.
    \item Das System muss gängige WordPress-Plugin-Best-Practices befolgen.
    \item Das System soll gängige Security-Best-Practices der WordPress-Developer-Ressourcen berücksichtigen.
\end{itemize}



\subsection{Funktionale Anforderungen}
\begin{itemize}
    \item Das System muss die Landingpage mit dem Gutenberg-Editor editierbar machen.
    \item Das System muss Blöcke bereitstellen, die die bestehende Landingpage abbilden können.
    \item Das System soll eine verbesserte Version des Dashboards bereitstellen.
    \item Das System soll eine bessere Übersicht für Nutzer:innen bieten.
    \item Das System soll eine verbesserte Möglichkeit bieten, die Front-End Views (z. B. E-Mail-Templates) zu stylen.
\end{itemize}

\section{Abgrenzung des Projektumfangs}
Zur Abgrenzung des Projektumfangs wurden jetzt hier die Anforderungen aufgebaut.
Ferner gilt es begleitend die schriftliche Ausarbetiung auszuführen.
Damit ein wissenschaftlicher konsens gefunden werden kann muss das ergebnis am ende evaluiert werden und das ergebnis kritisch betrachtet werden.
Ferner soll der weitere horizont für das zukünfitge vorgehen festgelegt werden und mögliche erweiterungen oder anpassungen angesprochen werden.

\chapter{Konzeption und Design}
\label{chap:concept}
Aufbauend auf der Analyse des Projektkontexts in Kapitel \ref{chap:context} werden in diesem Kapitel die konzeptionellen Überlegungen und das technische Design der Lösung vorgestellt. Ziel ist es, die identifizierten Schwachstellen zu adressieren und eine tragfähige Architektur für die Umsetzung des WordPress-Plugins zu entwerfen. Dabei werden sowohl die Struktur von Backend, Frontend und Helper-Layer als auch die Integration des Gutenberg-Editors konzeptionell beschrieben, um die Basis für die Implementierung in Kapitel \ref{chap:implementation} zu schaffen.
\section{Stakeholderanalyse und Nutzungskontext}
Zuerst werden die Stakeholder und der Nutzungskontext des Projekts Charigame beleuchtet.
Je nach Branche und Geschäftsmodell ergeben sich dabei unterschiedliche Anforderungen an die Funktionalität und den Einsatzkontext.
Eine Übersicht der Stakeholdergruppen ist in Abbildung~\ref{fig:stakeholder} veranschaulicht.
\begin{figure}[H]
    \centering
    \includesvg[width=0.75\textwidth]{images/stakeholder}
    \caption{Stakeholderanalyse Matrix (eigene Darstellung)}
    \label{fig:stakeholder}
\end{figure}

\begin{enumerate}
    \item \textbf{Dienstleistungsunternehmen}

Dienstleistungsunternehmen mit langfristigen Kundenbeziehungen fokussieren sich oft primär auf die Erhaltung ihrer bestehenden Geschäftspartnerschaften.
Hier geht es weniger um die Neukundengewinnung als mehr um die Pflege des bestehenden Kundenstamms.
Eine gamifizierte CSR-Lösung bietet die Möglichkeit die soziale Verantwortung sichtbar zu machen und die Bindung zu stärken.

    \item \textbf{Konzerne}

Großunternehmen nutzen CSR-Maßnahmen strategisch zur Stärkung von Geschäftspartnerschaften und zur Förderung des Markenimages.
Für diese Zielgruppe ist es wichtig, dass die Skalierbarkeit nahtlos und ohne Probleme geschieht.
Charigame kann hier einen innovativen Ansatz bieten, um Mitarbeitende, Partner und die Öffentlichkeit einzubinden.

    \item \textbf{Vermittler}

Unternehmen mit direktem Endkundenkontakt in wettbewerbsintensiven Märkten, profitieren von Alleinstellungsmerkmalen.
Charigame schafft durch seinen spielerischen Ansatz einen niedrigschwelligen Zugang zu CSR-Aktivitäten.
Dies kann die Kundenerfahrung emotional aufwerten, wodurch sich Wettbewerbsvorteile erschließen.

    \item \textbf{Produzenten}

Für produzierende Unternehmen bietet gamifizierte CSR die Möglichkeit, eine emotionale Bindung zwischen Endverbrauchern und der Marke herzustellen.\cite{zhang2025csr}
Spielerisch gestaltete Spendenaktionen lassen sich nahtlos in Marketingkampagnen integrieren.
Diese Aktionen stärken das Markenimage und erhöhen die Reichweite in digitalen Kanälen.\\\\
Zusammenfassend lässt sich festhalten, dass das Plugin \textit{Charigame} sowohl im \gls{b2b}- als auch im \gls{b2c}-Kontext unterschiedliche Mehrwerte schaffen kann.
Während im \gls{b2b}-Bereich die Aspekte Reputation und Kundenbindung im Vordergrund stehen, sind es im \gls{b2c}-Bereich vor allem Markenerlebnis und Differenzierung, die den Einsatz attraktiv machen.
\end{enumerate}

\section{Anforderungen an die Weiterentwicklung}
Aus den Erkenntnissen der Stakeholderanalyse und den identifizierten Schwächen des deskriptiven Stands, sowie der zu bearbeitenden These dieser Arbeit lassen sich diverse Anforderungen ableiten.
Diese Anforderungen wurden entsprechend dem definierten Projektumfang auf die wesentlichsten Aspekte fokussiert und werden nachfolgend adressiert.

% -------------------------------------------------
% Technische Anforderungen
\subsection{Technische Anforderungen}
\begin{enumerate}[label={[T\arabic*]},ref=T\arabic*]
    \item \label{T1} Das System muss die Abhängigkeit von der PRO-Version des Plugins ACF (Advanced Custom Fields) entfernen.
    \item \label{T2} Das System muss gängige WordPress-Plugin-Best-Practices befolgen.
    \item \label{T3} Das System soll gängige Security-Best-Practices der WordPress-Developer-Ressourcen berücksichtigen.
\end{enumerate}

% -------------------------------------------------
% Funktionale Anforderungen
\subsection{Funktionale Anforderungen}
\begin{enumerate}[label={[F\arabic*]},ref=F\arabic*]
    \item \label{F1} Das System muss die Landingpage mit dem Gutenberg-Editor editierbar machen.
    \item \label{F2} Das System muss Gutenberg-Blöcke bereitstellen, die die bestehende Landingpage abbilden können.
    \item \label{F3} Das System soll eine verbesserte Möglichkeit bieten, Inhalte zu bearbeiten.
    \item \label{F4} Das System kann eine verbesserte Version des Dashboards bereitstellen.
    \item \label{F5} Das System kann eine bessere Übersicht für Nutzer bieten.
\end{enumerate}

\section{Abgrenzung des Projektumfangs}
Zur Abgrenzung des Projektumfangs der Arbeit wurden die Anforderungen im Muss-Kann-Soll-Schema definiert.
Das bestehende Projekt bietet weiterführendes Potenzial wesentlich mehr Anforderungen zu definieren, die nicht abgedeckt wurden, um den Umfang im Rahmen zu halten.
Die definierten Anforderungen werden im weiteren Verlauf der Konzeption tiefergehend betrachtet, damit der angestrebte präskriptive Zustand erreicht wird.

\section{Plugin-Architektur}
Die Weiterentwicklung des Plugins \emph{Charigame} strebt eine modulare und wartungsfreundliche Pluginarchitektur an, wie einleitend in Kapitel \ref{chap:intro} skizziert.
Dabei wird das Ziel verfolgt, eine klare Trennung von Verantwortlichkeiten umzusetzen.
Ferner wird die Erweiterbarkeit und die Möglichkeit, das Projekt skalierbar zu entwickeln, ein wesentlicher Bestandteil der Architektur.
Grundlegend wird ein objektorientierter Ansatz verfolgt, um die Wiederverwendbarkeit des Codes zu gewährleisten.
Folgend werden die wichtigsten definierten Anforderungen im Zusammenhang mit den einzusetzenden Technologien adressiert.
\subsection{Einzusetzende Technologien}
\textbf{Carbon Fields}

Die Abhängigkeit des externen Plugins ACF Pro vollständig zu entfernen, geht aus der zuvor definierten Anforderung [\ref{T1}] hervor.
Anstelle von ACF Pro wird das Framework Carbon Fields genutzt, das über Composer eingebunden werden kann.
Carbon Fields ermöglicht die flexible Definition und Verwaltung von Metafeldern innerhalb von WordPress.
Im Vergleich zu ACF hat diese Lösung den Vorteil, dass sie als Bibliothek in das Plugin integriert werden kann.
Darüber hinaus ist Carbon Fields eine Open Source Lösung und es ist keine kostengebundene Lizenz notwendig.
\\\\
\textbf{Gutenberg Blockeditor}

Ein weiterer Kernbestandteil der Technologieauswahl ist die Integration von Gutenberg.
Dies erlaubt die Entwicklung maßgeschneiderter Blöcke für den Editor und stellt sicher, dass Inhalte direkt im Backend in einer Live-Vorschau bearbeitet werden können.
Für das Editor-UI wird React verwendet, da es die von WordPress empfohlene und im Gutenberg-Editor standardmäßig eingesetzte Technologie ist.
Die Kopplung von PHP im Backend und React im Editor ermöglicht eine klare Trennung von Daten- und Darstellungsschicht.
Durch den Einsatz dieser Lösung können die Anforderungen [\ref{F1}], [\ref{F2}] und [\ref{F3}] behandelt werden.
\\\\
\textbf{PHPCS}

Eine weitere zentrale Thematik der Weiterentwicklung bildet die Anforderung [\ref{T2}].
Unter Zuhilfenahme des \gls{phpcs} wird sichergestellt, dass die Wordpress Coding Standards konsequent eingehalten werden.
Diese Verwendung ermöglicht eine automatisierte Überprüfung des Codes, sodass Verstöße gegen die definierten Standards unmittelbar angezeigt werden.
Auf diese Weise wird nicht nur die Lesbarkeit, sondern auch die langfristige Wartbarkeit des Codes erheblich verbessert.
\\\\
Aus der geplanten Architektur und den eingesetzten Technologien geht weiterführend der Aufbau der Komponenten hervor.

\subsection{Aufbau der Komponenten}
Das Plugin soll eine zentrale Einstiegsklasse \texttt{Charigame} besitzen, die in der Hauptdatei des Plugins initialisiert wird.
Durch den definierten Plugin-Header sowie die Implementierung der WordPress-Hooks für Aktivierung und Deaktivierung wird diese Klasse geladen und stößt die Instanziierung aller weiteren benötigten Abhängigkeiten an.
Die Architektur sieht eine Reihe spezialisierter Klassen vor, welche die unterschiedlichen Funktionsbereiche des Plugins abbilden.
Dazu gehören unter anderem:

\begin{itemize}
    \item \textbf{Charigame Blocks:} Verantwortlich für die Registrierung und Bereitstellung der im Plugin genutzten Gutenberg-Blöcke.
    Darüber hinaus wird eine eigene Block-Kategorie eingeführt und die notwendigen Assets werden eingebunden.
    \item \textbf{Login Handler:} Kümmert sich um die Verwaltung der Loginsessions der Teilnehmer, prüft Gamecodes und stößt das Rendering der Landingpage an.
    \item \textbf{Color Manager:} Ermöglicht die Registrierung von Kampagnenfarben und deren dynamische Integration in das DOM.
    \item \textbf{Donation Manager:} Übernimmt die Logik der Spendenverteilung, speichert und aktualisiert Ergebnisse, berechnet Anteile und stellt diese für die weitere Verarbeitung bereit.
    \item \textbf{Email Sender:} Implementiert Funktionen zum E-Mail-Versand.
    Dazu gehören die Konfiguration des SMTP, das Versenden von Test- und Kampagnenmails, die Generierung des HTML-Templates sowie das Bereitstellen von Variablen.
    \item \textbf{Carbon Fields:} Verantwortlich für die Initialisierung und Bereitstellung des Frameworks Carbon Fields, das zur Verwaltung individueller Metafelder eingesetzt wird.
\end{itemize}

Ergänzend wird eine statische Konfigurationsdatei vorgesehen, die allgemeine Design-Mappings wie Abstände, Textausrichtungen oder Flexbox-Alignments enthält.
Diese statischen Definitionen basieren auf TailwindCSS und werden innerhalb der Gutenberg-Blöcke genutzt.
\newpage
\subsection{Custom Post Types}
Die zentrale Datenstruktur des Plugins wird über die Definition mehrerer Custom Post Types (\gls{CPT}) realisiert.
Folgende Post Types werden vorgesehen:

\begin{itemize}
    \item \textbf{Campaign:} Enthält alle zentralen Informationen einer Kampagne.
    Dazu gehören die Wahl des Spieltyps, spezifische Spieleinstellungen, Verlinkungen zu Landingpages und E-Mail-Templates sowie die Konfigurationen zur Spendenverteilung und der Zeitsteuerung.
    Zudem werden in der Campaign die Einstellungen für das Login-Formular hinterlegt.

    \item \textbf{Landingpage:} Stellt einen CPT zur Verfügung, der die Bearbeitung von Landingpages direkt im Gutenberg-Editor ermöglicht.
    \item \textbf{Game:} Dient als Zuordnungstyp für die verschiedenen im Plugin angebotenen Spiele.
    \item \textbf{Game-Settings:} Wird als neuer CPT erstellt, der die Spieleinstellungen aus der Kampagne nimmt und diese wiederverwendbar macht.
    \item \textbf{Recipient:} Enthält Informationen zu den begünstigten Empfängern einer Kampagne, darunter Name, Logo und Beschreibung.
    \item \textbf{User:} Speichert Teilnehmendeninformationen wie Vorname, Nachname, E-Mail-Adresse und Geburtsdatum.
    Außerdem werden hier Import- und E-Mail-Versand-Status dokumentiert.
    Die ursprüngliche Implementierung mit ACF wird durch Carbon Fields ersetzt.
    \item \textbf{Email Template:} Ermöglicht die Verwaltung von E-Mail-Templates innerhalb des Gutenberg-Editors.
\end{itemize}

Die verwendeten Custom Post Types, stehen wie in Abbildung~\ref{fig:datenmodell} visualisiert, in Relation zueinander.
Hier ist ersichtlich, dass die Campaign als zentraler Punkt die diversen Custom Post Types nutzt, um alle notwendigen Informationen für die Spendenkampagne zu aggregieren.
Zudem kann man erkennen, dass das geplante Dashboard die Daten aus der Kampagne und den Nutzern bezieht und diese durch weitere Parameter anreichert.
\begin{figure}[H]
    \centering
    \includesvg[width=1.0\textwidth]{images/datenmodell}
    \caption{Relation der Custom Post Types (eigene Darstellung)}
    \label{fig:datenmodell}
\end{figure}


\subsection{Verwaltungs- und Darstellungsebene}
Für den Administrationsbereich ist die Bereitstellung eines eigenen Dashboards geplant, das als zentrales Menü im WordPress-Backend integriert wird.
Dieses ersetzt die bisherige Lösung, die auf Data Tables basierte und stellt eine intuitivere Verwaltung der Kampagneninhalte bereit und adressiert die Anforderung [\ref{F4}] und [\ref{F5}].

Im Frontend-Bereich wird ein Template Loader eingesetzt, der die Darstellung der Campaign- und Landingpage-Inhalte übernimmt.
Zusätzlich steuert ein Asset Manager das Einbinden und Entfernen aller benötigten Skripte und Styles, wie beispielsweise backend.js.

Ergänzend zur Trennung von Frontend und Backend ist ein dedizierter Helper-Layer vorgesehen.
Dieser beinhaltet eine Helper-Komponente, die verschiedene serverseitige Funktionen bereitstellt.
Dazu gehören unter anderem die Registrierung von \gls{ajax}-Actions sowie öffentliche Methoden zur Abfrage von Spendenverteilungen oder zur Verwaltung aggregierter Kampagnenergebnisse.
Die Trennung in einen dedizierten Layer ermöglicht es, wiederkehrende Aufgaben zentral zu verwalten.
%\\\\
%\textbf{Datenmodell der Klassen}


%Geplante Struktur bietet bessere Wartbarkeit blabla
%Die Struktur erlaubt Redakteuren die Pflege mit vertrauten WP-Mechanismen; Ent-
%wickler profitieren von einer sauberen Trennung von Geschäftslogik und Präsentati-
%on.
%
%Dann gutenberg blabla
%Keine stanni blöcke, um den fokus auf die wesentlichen und erklären warum man weniger optionen bieten sollte
%Dann Farben und Abstände regulieren das ganze mit Tailwind als CSS, sodass keine eigene blockcss notwendig ist
%Blöcke maßgeblich statisch programmiert also mit allen möglichen elementen die der block vorgesehen hat aber alles anpassbar individualisierbar
%bei how-to auch das Paradigma der blöcke mit innerblocks umgesetzt.
%dann noch nicht aufnehmen game-settings
%Dann extra bonus block-design für Email TEmplate aufgebaut --

\chapter{Implementierung}
\label{chap:literature}
%
Fremdes Gedankengut muss immer kenntlich gemacht werden. Vor allem muss es überprüfbar und auffindbar sein. Hierzu dient die Technik des Zitierens und Belegens.
\par
Verschiedene Fachrichtungen und Studiengänge folgen spezifischen Zitierkonventionen. Wichtige Hinweise zu gängigen Zitationssystem und -stilen finden Sie in den E-Learning-Kursen des Schreibzentrums\footnote{\href{https://ilu.th-koeln.de/goto.php?target=cat\_52109\&client\_id=thkilu}{https://ilu.th-koeln.de/goto.php?target=cat\_52109\&client\_id=thkilu}}.
%
\section{Entwicklungsumgebung und Tools}
%Bitte legen Sie den Zitierstil immer am Anfang fest und wechseln sie ihn nicht.
\par
Wörtlich übernommene Textpassagen werden durch Anführungszeichen unten und oben (\enquote{\ldots}) kenntlich gemacht.
\par
Wenn Ihr Zitat bereits ein Zitat enthält, müssen Sie die \enquote{doppelten Anführungszeichen} im Text durch \enquote*{einfache Anführungszeichen} ersetzen. Dazu vergleiche auch Abschnitt~\ref{sec:specialCases}.
%Verwechseln Sie diese bitte nicht mit dem Apostroph oder dem Zoll-Zeichen auf Ihrer Tastatur. Die Tastaturkürzel für korrekte Anführungszeichen und andere Sonderzeichen sind bei Typefacts aufgelistet.
%
\subsection{Plugin-Softwarearchitektur}
Für alle Zitate muss ein Quellenverweis erstellt werden. Der Quellenverweis ist eine im Zitationsstil festgelegte Kurznotation, die auf die vollständige Literaturangabe im Literaturverzeichnis verweist. Quellenverweise können \emph{entweder} im laufenden Text (anglo-amerikanische Zitierweise bzw. Harvard-Stil) \emph{oder} über eine Fußnote am unteren Ende der Seite \emph{oder} in einer Endnote am Ende des gesamten Textes erfolgen. Hier gelten unterschiedliche fachliche Konventionen, die unbedingt beachtet werden müssen.
\par
Entscheidet man sich für Kurzverweise im Textfluss oder für Endnoten, kann der Fußnotenbereich für Kommentare und für Verweise auf Stellen im eigenen Text genutzt werden.
\par
\emph{Beachten Sie die Positionen von Hochzahl und Satzzeichen:}
\par
Wenn das wörtliche Zitat selbst mit einem Punkt endet, steht dieser vor dem beendenden Anführungszeichen. Die Hochzahl folgt dann direkt danach ohne Leerzeichen. Ein Punkt für den eigenen Satz entfällt dann.
\par
Wenn das wörtliche Zitat nicht mit einem Punkt endet gibt es zwei Fälle: Steht das Zitat mitten im eigenen Satz, folgt die Hochzahl direkt nach den Anführungszeichen. Steht das Zitat am Ende des eigenen Satzes, notiert man zuerst die Anführungszeichen, dann den eigenen Satzpunkt und erst dann die Hochzahl.
\par
Damit die Quellenverweise auch bei mehreren gleichlautenden Kurztiteln oder Jahreszahlen eines Autors eindeutig bleiben, erhalten die Jahreszahlen einen zusätzlichen kleinen Buchstaben.
%Oftmals ergibt sich dies erst gegen Ende der Arbeit. Daher kann es helfen, bei der Manuskripterstellung der Jahreszahl einen vorläufigen Kurztitel beizufügen. In der Endphase lässt sich dieser durch Suchen/Ersetzen über das Textverarbeitungsprogramm austauschen bzw. entfernen.
%
\subsection{Backend-Entwicklung}
Textverarbeitungsprogramme machen das Verweisen über \emph{derselbe} und \emph{ebenda} heute überflüssig, da man nicht mehr gezwungen ist, die gleichen Angaben wieder und wieder abzutippen. Sollten Sie sich (zum Beispiel auf Anraten Ihres Prüfenden) dennoch für dieses Verfahren entscheiden, empfehlen wir dringend, \emph{ebenda} und \emph{derselbe} etc. \emph{erst in der redaktionellen Endphase} einzufügen, weil die Bezüge erst dann klar sind.


\chapter{Fazit und Ausblick}
\label{ch:fazit-und-ausblick}

Ziel dieser Arbeit war die technische Weiterentwicklung des WordPress-Plugins \textit{Charigame}.
Im Detail galt es das Plugin zu einer modularen, wartbaren und redaktionell effizient nutzbaren Lösung durch den Einsatz des Gutenberg-Editors auszubauen.
Ausgangspunkt war eine historisch gewachsene Codebasis mit starker Abhängigkeit von ACF Pro sowie einer eingeschränkten Editierbarkeit von Inhalten.
Aufbauend auf den theoretischen Grundlagen, der Analyse des Projektkontexts und der konzeptionellen Ausarbeitung wurden in der Implementierung konkrete Maßnahmen umgesetzt.
Diese Maßnahmen haben die Architektur weiterentwickelt und zentrale Funktionalitäten überarbeitet.

\section{Zusammenfassung der Ergebnisse}
Die Weiterentwicklung führte zu Verbesserungen in mehreren Bereichen.
Im Aspekt der Architektur und Wartbarkeit wurde eine klare, modulare Plugin-Architektur mit objektorientierten Komponenten eingeführt.
Die Komponenten darunter Login-Handler, Color Manager und Donation Manager haben Funktionalitäten, die zuvor mit dem Frontend gekoppelt waren getrennt.
Dadurch konnten klare Verantwortlichkeiten gesetzt werden.
Darüber hinaus ist die zukünftige Erweiterbarkeit durch die klare Zuordnung der Klassen vereinfacht.

Das Entfernen externer Lizenzkosten erfolgte durch die Migration von ACF Pro zu Carbon Fields.
Bei der Gutenberg-Integration wurden eigene Gutenberg-Blöcke für die Sektionen der Landingpage entwickelt.

Standardkonformität und Sicherheit wurden durch systematische Anwendung der WordPress Coding Standards (PHPCS) gewährleistet.
Hier wurde konsequentes Input-Sanitizing und Output-Escaping sowie Nonce-Validierung relevanter AJAX-Endpunkte eingesetzt.

Der Datenzugriff und die Integration sind durch Freigabe relevanter Eigenschaften in der REST-API zur Editor-Integration vereinheitlicht.
Ferner wurde ein Caching von Kampagnenstatistiken eingeführt und die bestehenden Custom Post Types überarbeitet.

Im Bereich Administration und UX wurde ein modernisiertes Dashboard mit wiederverwendbaren Admin-Komponenten aufgebaut.

\section{Einordnung der Zielerreichung}
Die formulierten Zielsetzungen wurden erreicht und die konzipierten Anforderungen erfüllt.
Durch die Umsetzung der modularen Architektur, klar definierte Klassen und einer einheitlichen Namensgebung ist der Code übersichtlicher und nachhaltiger wartbar.
Mit PHPCS und der Ausrichtung an den WordPress-Guidelines wurden die Best Practices konsistent angewendet.
Die Gutenberg-Blöcke erlauben die Pflege aller Landingpage-Bereiche ohne Codeanpassungen.

\section{Reflexion und Grenzen}
Die Arbeit zeigt, dass die Verbindung aus WordPress-Standards, Block-Editor und modularer Architektur ein stabiles Fundament für WordPress Plugins bildet.
Die neue Codebasis reduziert Altlasten und schafft klarere Verantwortlichkeiten.
Die Redaktionserfahrung verbessert sich durch direkte Bearbeitung im Editor und eine geringere Abhängigkeit von Entwicklern bei Content-Änderungen.
Durchgängiges Sanitizing und Escaping, Nonces und Caching erhöhen Sicherheit und Performance.


Gleichzeitig bestehen Grenzen.\\
Das Dashboard wurde modernisiert, bietet aber Potenzial für tiefere Auswertungen wie Visualisierungen und Exporte.
Unit- und Integrationstests sowie CI/CD-Pipelines sind auszubauen, um eine Release-Sicherheit zu gewährleisten.
Die UI ist konsistent, doch ein systematisch dokumentiertes Designsystem kann helfen diese Konsistenz aufrechtzuerhalten.
\newpage
\section{Ausblick}
Auf Basis der Ergebnisse ergeben sich folgende mögliche Weiterentwicklungen:

\textbf{Dashboard und Analytics}: KPI-Visualisierungen mit tiefergehenden Auswertungsmöglichkeiten. Ebenso ein CSV/Excel-Export der Datengrundlage

\textbf{Qualitätssicherung}: Unit- und Integrationstests für PHP und JavaScript. Die Einrichtung einer CI/CD-Pipeline zwecks Verbesserung der Release-Sicherheit

\textbf{Internationalisierung}: Systematische i18n der Blöcke, Admin-Views und E-Mail-Templates ermöglichen die Mehrsprachigkeit

\textbf{Performance}: Erweiterte Objekt- und Transient-Caches, Asset-Bündelung und Lazy Loading steigern die Performance

\textbf{Barrierefreiheit}: WCAG-orientierte Überarbeitung, Komponentenbibliothek mit dokumentierten Patterns, verbessern die Zugänglichkeit

\textbf{Funktionalität}: Weitere Spieltypen, erweiterte Game-Settings als wiederverwendbare Presets erweitern den Funktionsumfang

\textbf{Integration}: Datenschutzkonforme Tracking-Konzepte, optionale Anbindung an externe CRMs

\section{Schlussbemerkung}
Mit der vorliegenden Weiterentwicklung wurde \textit{Charigame} auf eine solide, erweiterbare und standardkonforme Basis gestellt.
Die redaktionelle Arbeit im Gutenberg-Editor, die weiterentwickelte Architektur sowie sicherheits- und qualitätsorientierte Maßnahmen schaffen einen messbaren Mehrwert.


%
\printbibliography[prenote=mynote]
%
\appendix
\addchap{Anhang}
\begin{figure}[h!]
    \makebox[\textwidth][c]{%
        \includegraphics[width=1.3\textwidth]{block-structure}}
    \caption{Dateistruktur eines WordPress-Blocks \cite{wordpress2023}}
    \label{fig:block-structure}
\end{figure}
\begin{table}[h]
    \centering
    \renewcommand{\arraystretch}{1.3}
    \begin{tabular}{|p{3cm}|p{6cm}|p{5cm}|}
        \hline
        \textbf{Eingabefeld} & \textbf{Beschreibung} & \textbf{Eingabetyp} \\
        \hline
        Login-Form Überschrift & Definiert den Headline-Text der Login-Form & Texteingabe \\
        \hline
        Background Image & Hinterlegen eines Hintergrundbildes für die Login-Form & Bild-Upload \\
        \hline
        Company Logo & Eingabe für das Unternehmenslogo & Bild-Upload \\
        \hline
        Company Name & Name des Unternehmens & Texteingabe \\
        \hline
        Company Street & Straße des Unternehmens & Texteingabe \\
        \hline
        Company City & Stadt des Unternehmens & Texteingabe \\
        \hline
        Impressum Page & Referenz auf die Impressumsseite der Website & Seitenreferenz \\
        \hline
        Datenschutz Page & Referenz auf die Datenschutzseite der Website & Seitenreferenz \\
        \hline
        AGB Page & Referenz auf die AGB-Seite der Website & Seitenreferenz \\
        \hline
        Primary Color & Definition der primären \gls{ci}-Farbe & Farbeingabe (Hex-Wert) \\
        \hline
        Secondary Color & Definition der sekundären CI-Farbe & Farbeingabe (Hex-Wert) \\
        \hline
        Tertiary Color & Definition der tertiären CI-Farbe & Farbeingabe (Hex-Wert) \\
        \hline
        Section Primary & Festlegung eines primären Abschnittstrenners & Bild-Upload \\
        \hline
        Section Secondary & Festlegung eines sekundären Abschnittstrenners & Bild-Upload \\
        \hline
        Video \gls{mp4} & Hinterlegen eines Videos für die Landingpage & Medien-Upload          \\
        \hline
        Main-Font & Definition der Schriftart für die Landingpage (experimentell) & Medien-Upload \\
        \hline
    \end{tabular}
    \caption{Übersicht über konfigurierbare Eingabefelder der General Settings}
    \label{tab:eingabefelder_general_settings}
\end{table}

\begin{table}[h]
    \centering
    \renewcommand{\arraystretch}{1.3}
    \begin{tabular}{|p{3cm}|p{6cm}|p{5cm}|}
        \hline
        \textbf{Eingabefeld} & \textbf{Beschreibung} & \textbf{Eingabetyp} \\
        \hline
        Logo & Hinterlegen eines Bildes oder Logos für den Spendenempfänger & Bild-Upload \\
        \hline
        Name & Eingabe für den angezeigten Namen im Front-End & Texteingabe \\
        \hline
        Description & Festlegung des Beschreibungstextes für den Spendenempfänger  & Texteingabe \\
        \hline
    \end{tabular}
    \caption{Übersicht über konfigurierbare Eingabefelder der Donation Recipients}
    \label{tab:eingabefelder_donation_recipients}
\end{table}

\begin{table}[h]
    \centering
    \renewcommand{\arraystretch}{1.3}
    \begin{tabular}{|p{3cm}|p{6cm}|p{5cm}|}
        \hline
        \textbf{Eingabefeld} & \textbf{Beschreibung} & \textbf{Eingabetyp} \\
        \hline
        How To Play Headline & Definition der Überschrift der\newline Frontend Sektion & Texteingabe \\
        \hline
        Step Icon & Hinterlegen eines Icons oder Bildes für den Spielschritt & Media-Upload \\
        \hline
        Step Headline & Definition der Überschrift des Spielschritts & Texteingabe \\
        \hline
        Step Text & Festlegung des Beschreibungstextes für den Spielschritt  & Texteingabe \\
        \hline
        Step Color & Auswahl der Farbgebung (Primär-, Sekundär- oder Teritärfarbe)  & Auswahl \\
        \hline
    \end{tabular}
    \caption{Übersicht über konfigurierbare Eingabefelder der Game Types}
    \label{tab:eingabefelder_game_types}
\end{table}

\begin{table}[h]
    \centering
    \renewcommand{\arraystretch}{1.3}
    \begin{tabular}{|p{3cm}|p{6cm}|p{5cm}|}
        \hline
        \textbf{Eingabefeld} & \textbf{Beschreibung} & \textbf{Eingabetyp} \\
        \hline
        First Name & Definition des Vornamens des Teilnehmers & Texteingabe \\
        \hline
        Last Name & Definition des Nachnamens des Teilnehmers & Media-Upload \\
        \hline
        E-Mail & Hinterlegen der E-Mail-Adresse des Teilnehmers & Texteingabe \\
        \hline
        Next Birthday & Festlegung des Geburtstags für des Teilnehmers  & Datums-Auswahl\\
        \hline
        Imported & Auswahl, ob der Nutzer über das System importiert wurde & Checkbox \\
        \hline
        Email sent & Auswahl, ob eine Einladung an die Mail-Adresse gesendet wurde  & Checkbox \\
        \hline
    \end{tabular}
    \caption{Übersicht über konfigurierbare Eingabefelder der User}
    \label{tab:eingabefelder_users}
\end{table}

\begin{table}[h]
    \centering
    \renewcommand{\arraystretch}{1.3}
    \begin{tabular}{|p{3cm}|p{6cm}|p{5cm}|}
        \hline
        \textbf{Eingabefeld} & \textbf{Beschreibung}  \\
        \hline
        Email & Zeigt die Email-Adresse des Teilnehmers \\
        \hline
        Game Type & Gibt den Spieltyp an, der für die Kampagne festgelegt wurde \\
        \hline
        Game Code & Zeigt den Code an, mit dem der Teilnehmer sich im Spiel einloggen kann \\
        \hline
        Valid From & Stellt das Datum dar, ab welchem der Game Code gültig ist \\
        \hline
        Valid Until & Stellt das Datum dar, bis welchem der Game Code gültig ist  \\
        \hline
        Code Used & Gibt einen Zeitstempel an, wann der Game Code genutzt wurde um sich anzumelden \\
        \hline
        Last Played & Gibt einen Zeitstempel an, wann das Spiel zuletzt beendet wurde  \\
        \hline
        Highscore & Zeigt den erspielten Punktestand des Teilnehmers an \\
        \hline
        Recipient 1 & Stellt die Verteilung in Prozent dar, die der Teilnehmer für den Spendenempfägner 1 gewählt hat \\
        \hline
        Recipient 2 & Stellt die Verteilung in Prozent dar, die der Teilnehmer für den Spendenempfägner 2 gewählt hat \\
        \hline
        Recipient 3 & Stellt die Verteilung in Prozent dar, die der Teilnehmer für den Spendenempfägner 3 gewählt hat \\
        \hline
        E-Mail Sent & Zeigt durch 0 oder 1 an, ob die E-Mail an den Teilnehmer gesendet wurde  \\
        \hline

    \end{tabular}
    \caption{Übersicht über angezeigte Metriken der Data Table}
    \label{tab:metriken_datatable}
\end{table}
%TODO: FEHLER AUCH HIER IN DER ABBILDUNG BEI DEN HOW TO STEPS
\begin{figure}[H]
    \centering
    \includegraphics[width=0.66\textwidth]{images/legacy_landingpage_frontend}
    \caption{Gesamte Landingpage von Charigame im WordPress \\Frontend  (eigene Darstellung)}
    \label{fig:landing-frontend-legacy}
\end{figure}

\begin{figure}[H]
    \centering
    \includegraphics[width=1\textwidth]{images/legacy_verteilung_frontend}
    \caption{Spendenverteilungsseite von Charigame im WordPress Frontend (eigene Darstellung)}
    \label{fig:distribution-frontend-legacy}
\end{figure}

\begin{figure}[H]
    \centering
    \includegraphics[width=1\textwidth]{images/legacy_dankesseite_frontend}
    \caption{Dankesseite von Charigame im WordPress Frontend (eigene Darstellung)}
    \label{fig:dankesseite-frontend-legacy}
\end{figure}

\KOMAoptions{open=any}
%
\include{content/chapDeclaration}
%
%
\end{document}