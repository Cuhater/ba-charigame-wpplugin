%%%%%%%%%%%%%%%%%%%%%%%%%%%%%%%%%%%%%%%%%%%%%%%%%%%%%%%%%%%%%%%%%%%%%%%%%%%%%%%
% Begriffe für glossaries definieren
%%%%%%%%%%%%%%%%%%%%%%%%%%%%%%%%%%%%%%%%%%%%%%%%%%%%%%%%%%%%%%%%%%%%%%%%%%%%%%%

%=== Glossar ==================================================================
\newglossaryentry{plugin}{
	name={Plugin},
	description={Softwareerweiterung bestehend aus Codepaketen, die die
	Kernfunktionalität von WordPress erweitern. Plugins bestehen
	aus PHP-Code und können weitere Assets wie Bilder, CSS und
	JavaScript enthalten. \\(eigene Übersetzung nach \cite{wordpress2024plugin})}
}
\newglossaryentry{csr}{
    name={Corporate Social Responsibility},
    description={Unter "Corporate Social Responsibility" (CSR) ist die gesellschaftliche Verantwortung von Unternehmen im Sinne eines nachhaltigen Wirtschaftens zu verstehen.\cite{BMAS2022_CSRGrundlagen}}
}
\newglossaryentry{CPT}{
    name={Custom Post Type},
    description={Ein Custom Post Type ist ein individuell definierter Inhaltstyp in WordPress. Er wird über die WordPress Core Post API registriert und speichert die Daten in der posts-Tabelle der Datenbank. Ein Beispiel für einen Core-CPT ist der Datentyp „Seite“ (page).\cite{wordpress2025CPT}}
}
\newglossaryentry{cms}{
    name={Content-Management-System},
    description={Softwaresystem zur Erstellung, Verwaltung, Bearbeitung und Publikation von digitalen Inhalten wie Texten, Bildern, Videos und Multimedia-Dokumenten.}
}

%=== Abkürzungen ==============================================================
\newacronym{svm}{SVM}{support vector machine}

\newacronym{acf}{ACF PRO}{Advanced Custom Fields Pro}
\newacronym{ajax}{AJAX}{Asynchronous JavaScript and XML}
\newacronym{api}{API}{Application Programming Interface}
\newacronym{b2b}{B2B}{Business-to-Business}
\newacronym{b2c}{B2C}{Business-to-Consumer}
\newacronym{cdn}{CDN}{Content Delivery Network}
\newacronym{ci}{CI}{Corporate Identity}
\newacronym{cli}{CLI}{Command Line Interface}
\newacronym{cms}{CMS}{Content-Management-System}
\newacronym{crm}{CRM}{Customer-Relationship-Management}
\newacronym{csr}{CSR}{Corporate Social Responsibility}
\newacronym{csrf}{CSRF}{Cross-Site Request Forgery}
\newacronym{css}{CSS}{Cascading Style Sheets}
\newacronym{cta}{CTA}{Call to Action}
\newacronym{CPT}{CPT}{Custom Post Type}
\newacronym{gnu}{GNU}{GNU's Not Unix}
\newacronym{gpl}{GPL}{General Public License}
\newacronym{gsap}{GSAP}{GreenSock Animation Platform}
\newacronym{html}{HTML}{Hypertext Markup Language}
\newacronym{http}{HTTP}{Hypertext Transfer Protocol}
\newacronym{js}{JS}{JavaScript}
\newacronym{json}{JSON}{JavaScript Object Notation}
\newacronym{kpi}{KPI}{Key Performance Indicator}
\newacronym{mp4}{MP4}{Moving Picture Experts Group 4 Part 14}
\newacronym{mysql}{MySQL}{My Structured Query Language}
\newacronym{npm}{NPM}{Node Package Manager}
\newacronym{orm}{ORM}{Object-Relational Mapping}
\newacronym{php}{PHP}{Hypertext Preprocessor}
\newacronym{phpcs}{PHPCS}{PHP CodeSniffer}
\newacronym{rest}{REST}{Representational State Transfer}
\newacronym{smtp}{SMTP}{Simple Mail Transfer Protocol}
\newacronym{ssl}{SSL}{Secure Sockets Layer}
\newacronym{sql}{SQL}{Structured Query Language}
\newacronym{tls}{TLS}{Transport Layer Security}
\newacronym{uri}{URI}{Uniform Resource Identifier}
\newacronym{url}{URL}{Uniform Resource Locator}
\newacronym{wp}{WP}{WordPress}
\newacronym{xml}{XML}{Extensible Markup Language}
\newacronym{xsl}{XSL}{Extensible Stylesheet Language}
\newacronym{xss}{XSS}{Cross-Site-Scripting}

%=== Symbole ==================================================================
\newglossaryentry{sym:force}{
	name=\ensuremath{\vec{F}},
	description={Kraft, vektorielle Größe},
	type=symbols,
}
