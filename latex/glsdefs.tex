%%%%%%%%%%%%%%%%%%%%%%%%%%%%%%%%%%%%%%%%%%%%%%%%%%%%%%%%%%%%%%%%%%%%%%%%%%%%%%%
% Begriffe für glossaries definieren
%%%%%%%%%%%%%%%%%%%%%%%%%%%%%%%%%%%%%%%%%%%%%%%%%%%%%%%%%%%%%%%%%%%%%%%%%%%%%%%

%=== Glossar ==================================================================
\newglossaryentry{plugin}{
	name={Plugin},
	description={Softwareerweiterung bestehend aus Codepaketen, die die
	Kernfunktionalität von WordPress erweitern. Plugins bestehen
	aus PHP-Code und können weitere Assets wie Bilder, CSS und
	JavaScript enthalten. \\(eigene Übersetzung nach \cite{wordpress2024plugin})}
}
\newglossaryentry{wdr}{
	name={Wordpress Developer Resources},
	description={Ein online Handbuch, das Entwicklungsthemen in Wordpress abdeckt. Es wird kontinuierlich weiterentwickelt und gibt Entwicklern Definitionen, Coding Standards, Wordpress Core Ressources, Block Editor Ressources, gängige APIs, Beispiele und Tutorials an die Hand}
}
\newglossaryentry{csr}{
    name={Corporate Social Responsibility},
    description={Unter "Corporate Social Responsibility" (CSR) ist die gesellschaftliche Verantwortung von Unternehmen im Sinne eines nachhaltigen Wirtschaftens zu verstehen.\cite{BMAS2022_CSRGrundlagen}}
}
\newglossaryentry{dog}{
	name={Hund},
	description={Behaartes, vierbeiniges Säugetier. Bester Freund des
	Menschen}
}

%=== Abkürzungen ==============================================================
\newacronym{svm}{SVM}{support vector machine}

\newacronym{acf}{ACF PRO}{Advanced Custom Fields Pro}
\newacronym{api}{API}{Application Programming Interface}
\newacronym{cms}{CMS}{Content-Management-System}
\newacronym{csr}{CSR}{Corporate Social Responsibility}
\newacronym{css}{CSS}{Cascading Style Sheets}
\newacronym{gnu}{GNU}{GNU's Not Unix}
\newacronym{gpl}{GPL}{General Public License}
\newacronym{http}{HTTP}{Hypertext Transfer Protocol}
\newacronym{mysql}{MySQL}{My Structured Query Language}
\newacronym{php}{PHP}{Hypertext Preprocessor}
\newacronym{rest}{REST}{Representational State Transfer}
\newacronym{sql}{SQL}{Structured Query Language}

%=== Symbole ==================================================================
\newglossaryentry{sym:force}{
	name=\ensuremath{\vec{F}},
	description={Kraft, vektorielle Größe},
	type=symbols,
}
