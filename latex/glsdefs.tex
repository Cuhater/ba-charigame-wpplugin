%%%%%%%%%%%%%%%%%%%%%%%%%%%%%%%%%%%%%%%%%%%%%%%%%%%%%%%%%%%%%%%%%%%%%%%%%%%%%%%
% Begriffe für glossaries definieren
%%%%%%%%%%%%%%%%%%%%%%%%%%%%%%%%%%%%%%%%%%%%%%%%%%%%%%%%%%%%%%%%%%%%%%%%%%%%%%%

%=== Glossar ==================================================================
\newglossaryentry{plugin}{
	name={Plugin},
	description={Softwareerweiterung bestehend aus Codepaketen, die die
	Kernfunktionalität von WordPress erweitern. Plugins bestehen
	aus PHP-Code und können weitere Assets wie Bilder, CSS und
	JavaScript enthalten. \\(eigene Übersetzung nach \cite{wordpress2024plugin})}
}
\newglossaryentry{csr}{
    name={Corporate Social Responsibility},
    description={Unter "Corporate Social Responsibility" (CSR) ist die gesellschaftliche Verantwortung von Unternehmen im Sinne eines nachhaltigen Wirtschaftens zu verstehen.\cite{BMAS2022_CSRGrundlagen}}
}
\newglossaryentry{CPT}{
    name={Custom Post Type},
    description={Ein Custom Post Type ist ein individuell definierter Inhaltstyp in WordPress. Er wird über die WordPress Core Post API registriert und speichert die Daten in der posts-Tabelle der Datenbank. Ein Beispiel für einen Core-CPT ist der Datentyp „Seite“ (page).\cite{wordpress2025CPT}}
}
\newglossaryentry{cms}{
    name={Content-Management-System},
    description={Softwaresystem zur Erstellung, Verwaltung, Bearbeitung und Publikation von digitalen Inhalten wie Texten, Bildern, Videos und Multimedia-Dokumenten}
}
\newglossaryentry{blockeditor}{
    name={Block Editor (Gutenberg)},
    description={Seit WordPress 5 eingeführter Editor, der Inhalte aus modularen Blöcken zusammensetzt und direkt im Backend bearbeitbar macht}
}
\newglossaryentry{hook}{
    name={Hook},
    description={Erweiterungspunkt in WordPress, an dem eigener Code ausgeführt werden kann. Unterschieden wird zwischen Actions und Filtern}
}
\newglossaryentry{action}{
    name={Action},
    description={Hook-Typ in WordPress, der es erlaubt, zu bestimmten Ereignissen eigenen Code auszuführen (z.\,B. bei Plugin-Aktivierung)}
}
\newglossaryentry{filter}{
    name={Filter},
    description={Hook-Typ in WordPress, der Daten vor der Ausgabe verändert, indem Rückgabewerte durch eigene Funktionen gefiltert werden}
}
\newglossaryentry{shortcode}{
    name={Shortcode},
    description={Platzhalter in eckigen Klammern, der in Beiträgen/Seiten verwendet wird und beim Rendern durch dynamischen Inhalt ersetzt wird}
}
\newglossaryentry{enqueue}{
    name={Enqueue},
    description={Verfahren in WordPress zum korrekten Registrieren und Laden von Skripten und Styles (wp\_enqueue\_script, wp\_enqueue\_style)}
}
\newglossaryentry{wpcron}{
    name={WP-CRON},
    description={WordPress-eigener Pseudo-Cron zur zeitgesteuerten Ausführung von Aufgaben (Cron API), getriggert durch Seitenaufrufe}
}
\newglossaryentry{jquery}{
    name={jQuery},
    description={JavaScript-Bibliothek zur DOM-Manipulation und Eventbehandlung, häufig über ein CDN eingebunden}
}
\newglossaryentry{phpmailer}{
    name={PHPMailer},
    description={Bibliothek zum Versenden von E-Mails in PHP, in WordPress über wp\_mail nutzbar}
}
\newglossaryentry{threejs}{
    name={Three.js},
    description={JavaScript-3D-Bibliothek zur Darstellung von 3D-Grafiken im Browser}
}
\newglossaryentry{tailwind}{
    name={Tailwind CSS},
    description={Utility-First CSS-Framework, das über einen CLI/Build-Prozess Klassen generiert und in Projekten eingebunden wird}
}
\newglossaryentry{grunt}{
    name={Grunt},
    description={JavaScript-Task-Runner zur Automatisierung von Build- und Wartungsaufgaben}
}
\newglossaryentry{wplisttable}{
    name={WP List Table},
    description={WordPress-API zur Erstellung tabellarischer Listenansichten im Admin-Dashboard}
}
\newglossaryentry{templateinclude}{
    name={Template Include},
    description={Mechanismus in WordPress, um die geladene Template-Datei programmgesteuert zu überschreiben bzw. zu steuern}
}
\newglossaryentry{wpdb}{
    name={wpdb},
    description={WordPress-Datenbankschnittstelle (Klasse $wpdb$) für direkte SQL-Abfragen und Datenbankoperationen}
}
\newglossaryentry{dbdelta}{
    name={dbDelta},
    description={WordPress-Funktion zum Erstellen/Aktualisieren von Datenbanktabellen anhand einer Schema-Definition}
}
\newglossaryentry{carbonfields}{
    name={Carbon Fields},
    description={PHP-Bibliothek zur Definition von Feldern und Metadaten in WordPress, genutzt als Alternative zu ACF Pro}
}
\newglossaryentry{nonce}{
    name={Nonce},
    description={Einmal-Token in WordPress zur Absicherung von Formularen und Aktionen gegen Cross-Site Request Forgery}
}
\newglossaryentry{objectcache}{
    name={Object Cache},
    description={Caching-Schicht in WordPress zum Zwischenspeichern von Objekten und Datenbankergebnissen für bessere Performance}
}
%=== Abkürzungen ==============================================================
\newacronym{svm}{SVM}{support vector machine}

\newacronym{acf}{ACF PRO}{Advanced Custom Fields Pro}
\newacronym{ajax}{AJAX}{Asynchronous JavaScript and XML}
\newacronym{api}{API}{Application Programming Interface}
\newacronym{b2b}{B2B}{Business-to-Business}
\newacronym{b2c}{B2C}{Business-to-Consumer}
\newacronym{cdn}{CDN}{Content Delivery Network}
\newacronym{ci}{CI}{Corporate Identity}
\newacronym{cli}{CLI}{Command Line Interface}
\newacronym{cms}{CMS}{Content-Management-System}
\newacronym{crm}{CRM}{Customer-Relationship-Management}
\newacronym{csr}{CSR}{Corporate Social Responsibility}
\newacronym{csrf}{CSRF}{Cross-Site Request Forgery}
\newacronym{css}{CSS}{Cascading Style Sheets}
\newacronym{cta}{CTA}{Call to Action}
\newacronym{CPT}{CPT}{Custom Post Type}
\newacronym{gnu}{GNU}{GNU's Not Unix}
\newacronym{gpl}{GPL}{General Public License}
\newacronym{gsap}{GSAP}{GreenSock Animation Platform}
\newacronym{html}{HTML}{Hypertext Markup Language}
\newacronym{http}{HTTP}{Hypertext Transfer Protocol}
\newacronym{js}{JS}{JavaScript}
\newacronym{json}{JSON}{JavaScript Object Notation}
\newacronym{kpi}{KPI}{Key Performance Indicator}
\newacronym{mp4}{MP4}{Moving Picture Experts Group 4 Part 14}
\newacronym{mysql}{MySQL}{My Structured Query Language}
\newacronym{npm}{NPM}{Node Package Manager}
\newacronym{orm}{ORM}{Object-Relational Mapping}
\newacronym{php}{PHP}{Hypertext Preprocessor}
\newacronym{phpcs}{PHPCS}{PHP CodeSniffer}
\newacronym{rest}{REST}{Representational State Transfer}
\newacronym{smtp}{SMTP}{Simple Mail Transfer Protocol}
\newacronym{ssl}{SSL}{Secure Sockets Layer}
\newacronym{sql}{SQL}{Structured Query Language}
\newacronym{tls}{TLS}{Transport Layer Security}
\newacronym{uri}{URI}{Uniform Resource Identifier}
\newacronym{url}{URL}{Uniform Resource Locator}
\newacronym{wp}{WP}{WordPress}
\newacronym{xml}{XML}{Extensible Markup Language}
\newacronym{xsl}{XSL}{Extensible Stylesheet Language}
\newacronym{xss}{XSS}{Cross-Site-Scripting}

%=== Symbole ==================================================================
\newglossaryentry{sym:force}{
	name=\ensuremath{\vec{F}},
	description={Kraft, vektorielle Größe},
	type=symbols,
}
